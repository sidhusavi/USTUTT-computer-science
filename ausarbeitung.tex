% Dieses Dokument muss mit PDFLatex geestzt werden
% Vorteil: Grafiken koennen als jpg, png, ... verwendet werden
%          und die Links im Dokument sind auch gleich richtig
%
%Ermöglicht \\ bei der Titelseite (z.B. bei supervisor)
%Siehe https://github.com/latextemplates/uni-stuttgart-cs-cover/issues/4
\RequirePackage{kvoptions-patch}
%
\documentclass[
               paper=a4,
%               twoside, % fuer die Betrachtung am Schirm ungeschickt Optionen fuer typearea.
               BCOR1.92mm,DIV12,headinclude, %je höher der DIV-Wert, desto mehr geht auf eine Seite - Hack für BCOR. Bei BCOR2mm sind die Fuellpunkte beim Inhaltsverzeichnis falsch
               titlepage,
               bibliography=totoc,
%               idxtotoc,   %Index ins Inhaltsverzeichnis
%				liststotoc, %List of X ins Inhaltsverzeichnis, mit liststotocnumbered werden die Abbildungsverzeichnisse nummeriert
               headsepline,
               cleardoublepage=empty,
               parskip=half,
	       pointlessnumbers, %f"ur englische Texte, dann unten \ifdeutsch und \ifenglisch anpassen.
   %            draft    % um zu sehen, wo noch nachgebessert werden muss - wichtig, da Bindungskorrektur mit drin
               final   % ACHTUNG! - in pagestyle.tex noch Seitenstil anpassen
               ]{scrreprt}

%Englisch:			   
\let\ifdeutsch\iffalse
\let\ifenglisch\iftrue

%Deutsch:
%\let\ifdeutsch\iftrue
%\let\ifenglisch\iffalse

			   
\input{preambel/packages_and_options}

 %Der untere Rand darf "flattern"
\raggedbottom

%%%
% Wie tief wird das Inhaltsverzeichnis aufgeschlüsselt
% 0 --\chapter
% 1 --\section % fuer kuerzeres Inhaltsverzeichnis verwenden - oder minitoc benutzen
% 2 --\subsection
% 3 --\subsubsection
% 4 --\paragraph
\setcounter{tocdepth}{2}
%
%%%

\makeindex

%Angaben in die PDF-Infos uebernehmen
\makeatletter
\hypersetup{
            pdftitle={Machine Learning Methods for Fault Classification}, %Titel der Arbeit
            pdfauthor={Siddharth Sunil Gosavi}, %Author
            pdfkeywords={B8.1,I2.1,I2.6}, % CR-Klassifikation und ggf. weitere Stichworte
            pdfsubject={}
}
\makeatother

\begin{document}
%tex4ht-Konvertierung verschönern
\iftex4ht
% tell tex4ht to create picures also for formulas starting with '$'
% WARNING: a tex4ht run now takes forever!
\Configure{$}{\PicMath}{\EndPicMath}{} 
%$ % <- syntax highlighting fix for emacs
\Css{body {text-align:justify;}}

%conversion of .pdf to .png
\Configure{graphics*}  
         {pdf}  
         {\Needs{"convert \csname Gin@base\endcsname.pdf  
                               \csname Gin@base\endcsname.png"}%  
          \Picture[pict]{\csname Gin@base\endcsname.png}%  
         }  
\fi

%Tipp von http://goemonx.blogspot.de/2012/01/pdflatex-ligaturen-und-copynpaste.html
%siehe auch http://tex.stackexchange.com/questions/4397/make-ligatures-in-linux-libertine-copyable-and-searchable
%
%ONLY WORKS ON MiKTeX
%On other systems, download glyphtounicode.tex from http://pdftex.sarovar.org/misc/
%
\input glyphtounicode.tex
\pdfgentounicode=1




%\Coverpage
\VerbatimFootnotes %verbatim text in Fußnoten erlauben. Geht normalerweise nicht.

%\frontmatter
\input{macros/commands}
\pagenumbering{arabic}
%Eigener Seitenstil fuer die Kurzfassung und das Inhaltsverzeichnis
\deftripstyle{preamble}{}{}{}{}{}{\pagemark}
%Doku zu deftripstyle: scrguide.pdf
\pagestyle{preamble}
\renewcommand*{\chapterpagestyle}{preamble}

%Kurzfassung / abstract
%auch im Stil vom Inhaltsverzeichnis
\chapter*{Acknowledgments}
\addcontentsline{toc}{chapter}{Acknowledgments}
\begin{spacing}{1.5} 
First and foremost, I would like to express my sincere gratitude towards Prof. Dr. Hans-Joachim Wunderlich for providing me with the opportunity to contribute towards the research in fault classification techniques, with this thesis. His guidance and suggestions helped me throughout the research and writing of this thesis. I would like to thank my thesis supervisors - Ms. Laura Rodr\'{\i}guez G\'{o}mez and Mr. Alejandro Cook, for their guidance, patience and the motivation they provided all along, to meet the goals of the thesis. I cannot imagine better supervisors for my thesis. Without them, going though the thesis would not have been possible.

I am also thankful to all of the staff members of the institute for making my thesis at ITI an enjoyable experience.

My sincere thanks goes to all of my friends, in Germany, back home and abroad, for their constant encouragement and love.

Finally, I would like to thank my parents, Sou. Sheela Gosavi and Shri. Sunil Gosavi, and the rest of my family for their love and support not just during my thesis work, but also throughout my life.

\end{spacing}

\ifdeutsch
\chapter*{Kurzfassung}
\else
\chapter*{Abstract}
\addcontentsline{toc}{chapter}{Abstract}
\fi
\begin{spacing}{1.5} 
With the constant evolution and ever-increasing transistor densities in semiconductor technology, error rates are on the rise. Errors that occur on semiconductor chips can be attributed to permanent, transient or intermittent faults.

Out of these errors, once permanent errors appear, they do not go away and once intermittent faults appear on chips, the probability that they will occur again is high, making these two types of faults critical. Transient faults occur very rarely, making them non-critical. Incorrect classification during manufacturing tests in case of critical faults, may result in failure of the chip during operational lifetime or decrease in product quality, whereas discarding chips with non-critical faults may result in unnecessary yield loss.

Existing mechanisms to distinguish between the fault types are mostly rule-based, and as fault types start manifesting similarly as we move to lower technology nodes, these rules become obsolete over time. Hence, rules need to be updated every time the technology is changed.

Machine learning approaches have shown that the uncertainty can be compensated with previous experience. In our case, the ambiguity of classification rules can be compensated by storing past classification decisions and \emph{learn} from those for accurate classification. 

This thesis presents an effective solution to the problem of fault classification in VLSI chips using Support Vector Machine (SVM) based machine learning techniques.

\end{spacing}
\cleardoublepage


% BEGIN: Verzeichnisse

\iftex4ht
\else
\microtypesetup{protrusion=false}
\fi

%%%
% Literaturverzeichnis ins TOC mit aufnehmen, aber nur wenn nichts anderes mehr hilft!
% \addcontentsline{toc}{chapter}{Literaturverzeichnis}
%
% oder zB
%\addcontentsline{toc}{section}{Abkürzungsverzeichnis}
%\section*{Abkürzungsverzeichnis}
%
%%%

%Inhaltsverzeichnis anlegen
\tableofcontents

% Bei einem ungünstigen Seitenumbruch im Inhaltsverzeichnis, kann dieser mit
% \addtocontents{toc}{\protect\newpage}
% an der passenden Stelle im Fließtext erzwungen werden.

%listof* untereinandergesetzt
%ACHTUNG! Falls ein anderer Kapitelstil gewählt wird, muss der Code hier evtl.
%  angepasst werden
\begingroup 
\makeatletter
  \def\@makeschapterhead#1{%
  \vspace*{10\p@}%
  {\parindent \z@ \raggedright \reset@font
            \normalfont \vphantom{\@chapapp{} \thechapter}
        \par\nobreak\vspace*{10\p@}%
        \interlinepenalty\@M
    {\huge \bfseries %
	%
	%Default-Schrift: Serifenhaft (fuer englische Dokumente)
	% Dann sowohl A als auch B deaktivieren
    %A) Fuer serifenlose Schrift folgende Zeile aktivieren:
	\ifdeutsch
    \fontfamily{phv}\selectfont
	\fi
	%B) Fuer Kapitaelchen folgende Zeile aktivieren:
	%\fontseries{m}\fontshape{sc}\selectfont
	%
	#1\par\nobreak}
    %\vspace*{1\p@}%
\makebox[\textwidth]{\hrulefill}%    \hrulefill alone does not work
    \par\nobreak
    \vskip 5\p@
  }}
\makeatother
\let\cleardoublepage\clearpage
\listoffigures
\addcontentsline{toc}{chapter}{List of Figures}
\let\cleardoublepage\relax
\listoftables
\addcontentsline{toc}{chapter}{List of Tables}

%Wird nur bei Verwendung von der lstlisting-Umgebung mit dem "caption"-Parameter benoetigt
%\lstlistoflistings 
%ansonsten:
%\ifdeutsch
%\listof{Listing}{Verzeichnis der Listings}
%\else
%\listof{Listing}{List of Listings}
%\fi

%mittels \newfloat wurde die Algorithmus-Gleitumgebung definiert.
%Mit folgendem Befehl werden alle floats dieses Typs ausgegeben
%\ifdeutsch
%\listof{Algorithm}{Verzeichnis der Algorithmen}
%\else
%\listof{Algorithm}{List of Algorithms}
%\fi
\listofalgorithms %Ist nur für Algorithmen, die mittels \begin{algorithm} umschlossen werden, nötig
\addcontentsline{toc}{chapter}{List of Algorithms}
\endgroup

\cleardoublepage

\iftex4ht
\else
%Optischen Randausgleich und Grauwertkorrektur wieder aktivieren
\microtypesetup{protrusion=true}
\fi

% END: Verzeichnisse

%\pagenumbering{arabic}
\renewcommand*{\chapterpagestyle}{scrplain}
\pagestyle{scrheadings}
\input{preambel/pagestyle}
%
%
% ** Hier wird der Text eingebunden **
%
\chapter{Introduction}
\section{Motivation}
\section{Thesis Goals}

\section*{Thesis Organization}
Die Arbeit ist in folgender Weise gegliedert:
\begin{description}
\item[Kapitel~\ref{chap:k2} -- \nameref{chap:k2}:] Hier werden werden die Grundlagen dieser Arbeit beschrieben.
\item[Kapitel~\ref{chap:zusfas} -- \nameref{chap:zusfas}] fasst die Ergebnisse der Arbeit zusammen und stellt Anknüpfungspunkte vor.
\end{description}

%Die Angabe des schlauen Spruchs auf diesem Wege funtioniert nur,
%wenn keine Änderung des Kapitels mittels den in preambel/chapterheads.tex
%vorgeschlagenen Möglichkeiten durchgeführt wurde.
\chapter{Faults in VLSI Systems}
\label{chap:chapter2}
%\vspace{-3cm}
%\vspace{2cm}

The reliability is always a cause of concern during chip manufacturing. A manufactured chip needs to function correctly not just during post-manufacturing tests but during the complete lifespan of the final product. The typical lifespan for a chip designed for commercial purpose is defined as 11.4 years or 100,000 hours \cite{kishore2009}. The failure rate of ICs with respect to time is shown in figure~\ref{fig:bathtubcurve}, typically known as the \emph{bathtub curve}.

\begin{figure}[h]
  \begin{center}
    \captionsetup{justification=centering}
    \includegraphics[scale=0.5]{figures/bathtubcurve.png}
    \caption{Bathtub Curve}
    \label{fig:bathtubcurve}
  \end{center}
\end{figure}

The first region of the graph is called early failures or \emph{infant mortality region}. The second region is the lifetime of the device when random failures occur. The error rate in this region is low and constant. The third region of the graph is wear-out and is caused by failures at the end of the useful life \cite{kishore2009}. It can be expected that the ICs will not enter this region due to technology advances and obsolescence. This makes the first region important from the view of product quality. Early failures of chips can be remedied to a great extent by testing them immediately after manufacturing \cite{Agrawal2000} and thus it is important to detect faults at the manufacturing level.

Failure of a semiconductor chip can be described using a fault, defect or an error. A \emph{defect} in an electronic system is an unintended difference between implemented hardware and its design \cite{Agrawal2000}. Defects can occur either during the manufacturing process or lifetime of the device. An \emph{error} is said to have occurred when an unintended signal is generated by the system. An error may result in failure at primary outputs of the system. An error is essentially the manifestation of a defect. For the purpose of analysis, a defect is modeled as a \emph{fault}, which is simply representation of defect at the abstracted function level.

\begin{figure}[h]
  \begin{center}
    \captionsetup{justification=centering}
    \includegraphics[scale=0.75]{figures/vlsitesting.png}
    \caption{Typical test flow for VLSI Chips}
    \label{fig:vlsitesting}
  \end{center}
\end{figure}

The figure~\ref{fig:vlsitesting} shows how faulty chips are identified. To decide whether the Device Under Test (DUT) is working properly, a set of input stimuli called \emph{test pattern set} is applied to the DUT. The \emph{output response} is observed and is then compared to the standard output. If these outputs do not match then the chip is said to be faulty.

The sources of the fault can be either internal or external. When healthy chips fail due external mechanisms like $\alpha$-particle strikes, they are thrown away in a typical test, which contributes to yield loss. Thus to maintain product quality and to reduce yield loss, it is important classify the faults according their criticality. Section 2 of this chapter describes a taxonomy for such classification according to their sources and characteristics. It also focuses on various fault models that can be used to analyze these faults. The existing techniques for such fault classification are explained in section 3. The last section of this chapter explains diagnostic techniques and how they can used to classify faults.

\section{Fault Taxonomy}
\label{sec:secft}
According to the source and behavior of faults, they can be classified into three types: \emph{permanent}, \emph{intermittent}, and \emph{transient} \cite{Constantinescu2003}. Permanent faults reflect irreversible physical changes. Intermittent faults occur because of an unstable or a marginal hardware, and can be activated by environmental changes, like higher or lower temperatures and voltage. Transients occur because of temporary environmental conditions \cite{Constantinescu2003}. The likelihood of these faults is expected to increase with increase in transistor densities on semiconductor chips \cite{Constantinescu2007}.

\subsection{Permanent Faults}
Permanent faults are those which occur due to physical defects on the chip. The source of these faults generally lies in the issues of manufacturing process. However, they can also occur during operational lifetime of the circuit, especially when circuit is old and starts to wear-out.

The common sources of permanent faults are described below:
\begin{description}
\item[Manufacturing process:] Sso-called \emph{spot defects} can occur during manufacturing of a VLSI chip, and take form of either missing or extra material. Such a defect can cause an unwanted short or open between nodes or make an unintended multi-terminal transistor,leading to changed circuit topology. These defects mainly arise from some contamination, usually in form of dust particles or liquid droplets deposited on the wafer surface during some fabrication step \cite{Khare1996}. Also missing or excess metal may cause unwanted capacitance and resistance respectively resulting in delay lines \cite{Wagner1995}.

\item[Wear-out:] \emph{Electromigration} (EM) is defined as mass transport of metal atoms created by collision of electrons \cite{Ghate1982}. This movement of material will result in voids or hillock growth as in figure~\ref{fig:mfgdefects}, which can result in an open circuit or a short between adjacent tracks \cite{AnalogDevices2000}. With lower technology nodes, the wire widths are also getting smaller making EM a serious problem.
\end{description}

\begin{figure}[h]
  \begin{center}
    \captionsetup{justification=centering}
    \includegraphics[scale=0.75]{figures/mfgdefects.png}
    \caption{Manufacturing defects as sources for intermittent and permanent faults \cite{Lehtonen2009}.}
    \label{fig:mfgdefects}
  \end{center}
\end{figure}

Once a permanent fault appears in the system, it does not go away, unless the offending component is replaced. They are localized on the chip, and hence will affect the same set of primary outputs (POs). Permanent faults are reproducible and provide a predictable output response. Permanent faults only go away once the offending component is replaced \cite{Constantinescu2007}.

Some of the ways to model permanent faults are noted below, these are the models used to generate experimental data in this work.
\begin{description}
\item[Stuck-at fault model:]
Stuck-at fault models are the most simplest fault models. Due to its simplicity it is a widely used fault model\cite{Larsson2006}. It assumes that the fault location has a fixed logical value, either stuck-at 0 or stuck-at 1. These can be seen as short to ground or short to power supply respectively. When it is assumed that there is only one fault in the circuit at a time then \emph{single stuck-at} (SSA) model is used, otherwise in case of multiple defects \emph{multiple stuck-at} model is used.

\item[Wired AND/OR fault model:]
Unlike stuck-at, \emph{bridge fault} models a short between signal lines. \emph{Wired AND/OR} fault model is a type of the bridging fault model. These models are used to describe the logic behavior of two nodes that are shorted in the circuit. Wired AND model assumes that the faulty node of the bridge always has value 0, whereas wired OR model assumes faulty node has value 1. 

\item[Delay fault model:]
\emph{Delay fault} model is used to model timing related faults. Delay testing is required for modern VLSI systems running at high frequencies, as even minor timing violations can lead to system performing out of specifications \cite{Larsson2006}. There are two was to realize delay viz. \emph{Gate delay} and \emph{Path delay} models. Gate delay model assumes that the delay is only between input and output of individual logical gates on chip. In contrast, path delay models assume that the delay is spread over complete path from input to output. 
\end{description}
\subsection{Intermittent faults}
\label{sec:secif}
Intermittent faults are those caused by a marginal or an unstable hardware and are activated when certain conditions like voltage, temperature or frequency are met \cite{Constantinescu2003, Lehtonen2009}. Intermittent faults often precede occurrence of permanent failures \cite{Lehtonen2009}.

Some of the common sources for intermittent faults are described below:
\begin{description}
\item[Manufacturing defects:] As illustrated in figure~\ref{fig:mfgdefects} \emph{metal silvers} are stray pieces of metal on die due to some process imperfections. In certain conditions like increase in temperature, the metal may expand and touch the interconnects creating a short. In some cases the short might cause a current surge, damaging the circuit and can manifest into a permanent fault \cite{Hawkins2003}. Similarly cracks, as shown in same figure can continue to work normally at design temperature but at low temperatures can cause open circuits.

\item[Technology scaling:] With the technology scaling, the reduced thickness of oxide layers may result in current leakage, with a mechanism known as soft breakdown (SBD) \cite{Stathis2001}. In such a breakdown, the current fluctuates creating intermittent fault, without causing a thermal damage \cite{Stathis2001, Constantinescu2007a, Constantinescu2007}. 

\item[Power droop:] With the technology scaling, supply voltages are also lowered down, this results in degraded tolerance to power supply noise resulting in a \emph{low frequency power droop}\cite{Polian2007}. A \emph{high frequency power droop} occurs when multiple cells on a chip connected to the same power grid segment switch in the same direction, increasing their current demand causing a power starvation in some other part of chip \cite{Polian2007}. 
\end{description}

Once intermittent fault appears in the system, its probability of recurrence increases with time\cite{Bondavalli2000}. These faults are localized on the chip. They will affect the same set of primary outputs (POs) on recurrence. Intermittent faults have tendency to occur in bursts \cite{Constantinescu2007, Constantinescu2003}. Intermittent faults are not reproducible every time for same set of test patterns. Intermittent faults only go away once the offending component is replaced \cite{Constantinescu2007}.

Intermittent faults can be modeled as conditional stuck-at faults, activated by trigger condition. The activation condition can be expressed as a Boolean function and can depend on timing or environmental conditions \cite{Holst2009}.

\begin{description}
\item[High frequency power droop:] This type of fault occurs when specific set of input causes power starvation in some other part of the chip. Hence, this fault is modeled as a set of aggressor lines $a_1,a_2,...$ and a victim line $v$ \cite{Polian2007}. The fault occurs on victim line, due to presence of the aggressor lines.
\end{description}

\subsection{Transient faults}
\label{sec:sectf}
Transient faults are  deviations of normal circuit function caused by some  environmental factors or some external phenomenon. They are called soft errors as they do not do any permanent damage to the chip. A \emph{single-event upset} (SEU), which is change in value of single bit, is the most common manifestation of transient faults.


Some of common sources for transient faults are noted below:
\begin{description}
\item [ESD and EMI:] \emph{Electromagnetic interference} caused by sources emitting high energy signal may interfere with the working chip to bring about SEUs. An \emph{electrostatic discharge} due to users releasing static charge can also affect chips to cause transient faults.

\item[Particle strikes:] When an $\alpha$-particle, a proton or a neutron passes through a semiconductor material and starts to loose energy, it frees electron-hole pairs along its path \cite{Dodd2003}. If this material happens to be a reversed biased p-n junction, it can result in significant transient currents to bring about an SEU. Hence with scaling to lower technology nodes, it is very likely that probability of such SEUs will increase.

\end{description}


Transient faults are non-deterministic faults. These faults are not localized hence can affect any of the POs. Transient faults are not reproducible. Replacing the offending component may not make transient faults to go away \cite{Constantinescu2007}. Transient faults are isolated incidences of error occurrence. They usually do not occur in bursts like intermittent faults \cite{Constantinescu2007, Constantinescu2003}. Once the condition triggering the transient fault disappears, circuit return to normal functioning.

One of the ways  to simulate a transient fault is to implement a conditional stuck-at fault, at multiple fault location and to use a deterministic function to trigger the fault. This work uses a bit-flip as a fault model to model transient faults.

\begin{description}
	\item[bit-flip] A bit at random location is flipped using a trigger condition. This corrupted in value is then propagated through circuit \cite{Gracia2001}. 
\end{description}

\section{Fault Classification}
\label{sec:secfc}
Test flows like the one shown in figure~\ref{fig:vlsitesting} are able to distinguish between faulty and healthy chips. The traditional manufacturing tests still work the way as shown in figure~\ref{fig:traditionaltestflow} \cite{Weste1985}. Chips are tested for faulty behavior, and all chips showing single fault in a test run are thrown away.

\begin{figure}[h]
  \begin{center}
    \captionsetup{justification=centering}
    \includegraphics[scale=0.75]{figures/traditionaltestflow.png}
    \caption{Traditional flow for fault classification}
    \label{fig:traditionaltestflow}
  \end{center}
\end{figure}

With the advances in manufacturing processes, the number of permanent faults is increasing \cite{kishore2009}. However, on the other hand the impact of soft-errors and other non critical failures is increasing at much higher rate\cite{Constantinescu2003}. Some studies have indicated that up to 80\% failures can be attributed to SEUs \cite{Iyer1986, Dharchoudhury1994, kishore2009}, which suggest that these cases will result in an unnecessary yield loss. This section describes a few techniques to discriminate between different error categories described earlier in the chapter. As the fault taxonomy we used is similar in characteristics to those observed on PCs or workstations, the approaches used for their classification also provide a few pointers towards fault classification in VLSI systems. 

In \cite{Lin1990} a mechanism to classify between transient and intermittent faults is explained for error log analysis. In a technique called \emph{Dispersion Frame Technique} (DFT), the inter-arrival time in between successive error events of same error types is used to determine type of the fault in the system. Heuristics are applied to determine their closeness in time and affected area, which are then considered as parameters to decide whether the error is of the same type.

Authors in \cite{Iyer1990} use a similar technique to identify persistent failures in the system. Here they have used error rates to build up correlation using simple probabilistic techniques between error records, leading to a set of symptoms which may suggest a common cause (permanent errors).

A probabilistic approach is considered in \cite{Pizza1998}, which updates the probability of module being affected by permanent fault. It then weighs the consequences of actions performed by a faulty module against a fault-free module. It uses Bayesian inference to discriminate between permanent and transient errors

Historically a lot of work has been done to analyze impact of different types of faults on VLSI systems \cite{Constantinescu2003,Constantinescu2007,Dodd2003} and to classify them \cite{Savir1980, Espinosa2013, Bondavalli2000, DeKleer2009}.

The most popular techniques to classify transients from other types of faults are grouped under a family called $\alpha$-count techniques \cite{Bondavalli2000}. In a scheme called \emph{single threshold $\alpha$-count techniques}, a single threshold is established and if error count exceeds this threshold then the fault is classified as permanent or intermittent, whereas a smaller non-zero value indicates presence of transient faults. In an other variant of the same called \emph{dual threshold $\alpha$-count techniques}, two thresholds are established. If the error count exceeds first threshold then that component is assigned a restricted functionality and when it exceeds the second threshold it is taken out of service, like the single threshold technique. However, a component which is in-between thresholds can be taken in full service once its error count is lowered than first threshold.

Another approach using diagnostic probabilities in \cite{DeKleer2009} is able to distinguish permanent faults from faults with non-deterministic behavior.

\section{Fault Diagnosis}
\label{sec:secfd}
\emph{Diagnosis} is the process of locating faults in a physical chip at the various levels down to real defects. In the traditional fault-dictionary based diagnosis, we are given two sets of data, a \emph{predicted output} $P$ which is a set of outputs when fault a particular fault is active in the system, a \emph{measured set} $M$, which is the observed fault behavior and corresponding fault $ f_i \in \{f_1,f_2,...,f_n\} $. When the two sets match i.e. $P = M$ the corresponding fault is diagnosed to be active . When $P \neq M$ then logic diagnosis tries to find the best fitting explanation. However it is practically infeasible to construct such fault dictionaries for modern circuits, as the fault dictionary should consist of all possible faults and their combinations \cite{Wang2010}. 

An adaptive approach which does not use fault dictionaries called \emph{Partially Overlapping Impact couNTER} (POINTER) for diagnosis is described in \cite{Holst2009}. This approach uses test pattern sets with the \emph{Single Location At a Time} (SLAT) property \cite{Bartenstein2001} to diagnose faults present in the circuit. The authors in \cite{Holst2009} define a \emph(Fault Machine) (FM), \emph{i.e.} a reference circuit with stuck-at faults injected. As shown in figure~\ref{fig:evidance}, a tuple of parameters called \emph{evidence} is defined as,

\[e(f,T) = \{\sigma_T, \iota_T, \tau_T, \gamma_T\}\]. 

\begin{description}
	\item[$\sigma_T$] is the sum of number of failing output  where the Device Under Diagnosis (DUD) and the FM match. It is calculated as sum of all $\sigma_t$ values by injecting one stuck-at fault at a time in the FM.
	\item[$\iota_T$] is the sum of number of output pins which fail in the FM but are correct in the DUD. It is calculated as sum of all $\iota_t$ values by injecting one stuck-at fault at a time in the FM.
	\item[$\tau_T$] is the sum of number of output pins which fail in the DUD but are correct in Fthe M. It is calculated as sum of all $\tau_t$ values by injecting one stuck-at fault at a time in the FM.
 	\item[$\gamma_T$] is the sum of maximum of corresponding values of $\iota_t $ and $\tau_t$ for every test pattern.
\end{description}

\begin{figure}[h]
  \begin{center}
    \captionsetup{justification=centering}
    \includegraphics[scale=1.00]{figures/evidance.png}
    \caption{Evidence generated during diagnosis \cite{Holst2009}}
    \label{fig:evidance}
  \end{center}
\end{figure}

The values of $\iota$ and $\tau$ are high and $\sigma$ is low or zero, if the fault present in DUT is not explainable using any of the stuck-at faults injected in FM. Hence $\gamma$ is an indication about the ability to explain of the fault. For a permanent fault, the value of gamma would be low (zero). 

After diagnosis, in a process called \emph{ranking} obtained values are sorted as evidence with lowest value of $\gamma$ first. The evidences with the equal values of $\gamma$ are then sorted with the highest value of $\sigma$ first. For evidences with equal values of both $\sigma$ and $\gamma$, the ranking algorithm puts evidences with lowest value of $\iota$ first \cite{Holst2009}. This process brings the most explainable faults first. The result of diagnosis is then returned as the evidence with the highest rank.

These parameters vary depending on the type of fault present in the circuit. For permanent faults, the value of evidence parameters remains same, when same test pattern is applied at the input. The evidence would vary for each test run, as intermittent and transient faults are non-deterministic.




%Die Angabe des schlauen Spruchs auf diesem Wege funtioniert nur,
%wenn keine Änderung des Kapitels mittels den in preambel/chapterheads.tex
%vorgeschlagenen Möglichkeiten durchgeführt wurde.
\chapter{Machine Learning}
\label{chap:chapter3}
%\vspace{-3cm}
%\vspace{2cm}
An \emph{algorithm} is set of instructions used to convert input values to output, based on certain rules. Consider an example where we need to find all even numbers from dataset. Here, we can set up a \emph{rule} that if number is completely divisible by two then it should be included in the output dataset, otherwise not. Naturally, as there can be more than one way to solve a problem, there can be more than one algorithm to solve it. However there are certain examples where formation of set of rule is practically infeasible. For example, consider handwriting recognition software used to scan handwritten forms. Figure illustrates problem at hand, where a simple character can be written in a number of ways. It is interesting to note that humans are able to read this data without trouble, but it is really difficult express a certain rules which will result in accurate recognition with help of an algorithm. Machine learning is employed in such cases. Specifically \emph{Machine Learning} (ML) is programming computers to optimize a performance criterion (e.g. character recognition) using example data or past experience \cite{Alpaydin2004}. 

\begin{figure}[h]
  \begin{center}
    \captionsetup{justification=centering}
    \includegraphics[scale=0.5]{figures/charrec.png}
    \caption{Example of Machine Learning: Character recognition}
    \label{fig:charrec}
  \end{center}
\end{figure}

The "example data" with its \emph{label} is collectively called as \emph{training set}, and it is used to teach machine learning how the character with given label looks like, so that ML can recognize when it encounters similar data in future. Machine learning can be applied in wide range of applications where it is not possible to express expertise but a large amount of sample data is available. Typical applications of machine learning include computer vision, pattern recognition, spam filtering, search result optimization etc. 

\section{Types of learning algorithms}
Based on application, ML algorithms can be can be classified in two major categories \emph{viz.} supervised learning and unsupervised learning. 

\emph{Supervised learning} algorithms are used when labels of the data to be are known. A spam filter is a good example where supervised learning can be used for \emph{classification}. Here we know an email received is either "spam" or "not-spam", these categories can be used as labels for the sample population and learning algorithm can classify within these two type.  One more application of supervised learning is to predict a numerical value in \emph{regression}. Consider a problem to predict value of a used property, the input parameters in this case are initial value, year of construction, size of property, locality and so on, whereas output is current resale value. one can construct a training set of known resale values and receptive values of input parameters and train leaning algorithm to predict other inputs. To generalize, aim in supervised learning is to learn mapping from input to output whose correct vales are provided by supervisor \cite{Alpaydin2004}.

\emph{Unsupervised learning} or \emph{clustering} is used in classification problems where the labels for the data are not known. An example of such problem is document clustering \cite{Alpaydin2004}. One of applications of document clustering is to cluster news reports which belong to same category like sport, science, art and so on. The number of such categories is not clear, and the machine learning application in such case needs to cluster articles based on some common words, and provide the supervisor data, which he may use to label clustered groups.

In case of fault classification, we have clearly defined taxonomy in earlier chapter, making our case as supervised classification problem. In following subsection, we define basic terms as applied to case of supervised learning.

\section{Basic concepts in machine learning}

A \emph{feature} $(x_i)$ is a result of measurement made on a unit input data. Generally, a set of features $(\boldsymbol{x}^t)$ is needed to characterize a unit of input data and is expressed as,
\[ \boldsymbol{x}^t = \left[ x_1, x_2, \ldots x_m \right]^T \]  
Its label $r$ denotes the class $C_i \in \{C_1, C_2 \ldots C_k\}$ it belongs to and is denoted as,
\[ r_i^t = \left\{ \begin{array}{ll}
         1 & \mbox{if $\boldsymbol{x}^t \in C_i$};\\
         0 & \mbox{if $\boldsymbol{x}^t \in C_j, j \neq i$}\end{array} \right. \] 
The training set $X$ is then defined as ordered set containing $N$ values of such examples,
\[ X = \{\boldsymbol{x}^t , \boldsymbol{r}^t \}_{t=1}^N  \]
The aim for machine learning algorithm is to learn values in training set and then classify new examples $\boldsymbol{x}$ by estimating value of $C(\boldsymbol{x})$. To achieve this, the algorithm tries to find out a hypotheses $h_i, i \in\{1,2, \ldots k\}$ from a set of all possible hypotheses such that,
\[ h_i(\boldsymbol{x}) = \left\{ \begin{array}{ll}
         1 & \mbox{if $\boldsymbol{x}^t \in C_i$};\\
         0 & \mbox{$\boldsymbol{x}^t \in C_j, j \neq i$}\end{array} \right. \] 
The \emph{empirical error} after training is calculated as,
\[ E(\{h\}_{i=1}^k|X) = \sum\limits_{t=1}^N \sum\limits_{i=1}^k | h_i(\boldsymbol{x}) \neq r_i^t ) | \]

Figure~\ref{fig:mlfitting} shows two possible hypotheses $h_1$ and $h_2$ for a simple 2-class classification problem, both with same value of empirical error and also actual boundary of classification $C$. If we choose hypothesis $h_1$ then the examples which lie in region between $h_1$ and $C$ will get incorrectly classified and this is called as \emph{overfitting}. On the other hand, if we choose $h_2$ then same will happen for examples in region between $C$ and $h_2$, called \emph{underfitting}. To avoid this and to get a hypothesis which is as close to $C$ as possible, typically one more labeled dataset with examples other than training set called as \emph{cross-validation set} is picked. The empirical error is then calculated over this set and hypotheses obtained during training and the hypothesis with least value of error is then selected. 

\begin{figure}[h]
  \begin{center}
    \captionsetup{justification=centering}
    \includegraphics[scale=0.45]{figures/mlfitting.png}
    \caption{Example of overfitting and underfitting}
    \label{fig:mlfitting}
  \end{center}
\end{figure}

We use the term \emph{sample population} collectively for training set and cross-validation set.

\section{Machine Learning Algorithms for Classification}
\label{sec:c3mlclassification}
This section provides a brief overview of some of the most frequently used machine learning algorithms for classification problem. It includes advantages and disadvantages for individual algorithms.

\subsection{Na\"{i}ve Bayes}
Na\"{i}ve Bayes is a type of Bayesian classifier. It is one of most simple algorithms for learning, however in some cases it may outperform most sophisticated learning algorithms \cite{John1995}. Na\"{i}ve Bayes uses maximum-likelihood estimation to classify new examples. It is based on Bayesian theorem which states,
\[ posterior = \frac{likelihood \times prior}{evidance} \]
During training of Na\"{i}ve Bayes, probability of each class is calculated and stored as prior probability for that class. It also calculates probability for instances $\boldsymbol{x}$ given its class $c$. Under the assumption that attributes in $\boldsymbol{x}$ are independent, it simply becomes product of probabilities of each single attribute \cite{Williams2006}.
Hence Bayesian theorem, when applied to classification problem using Na\"{i}ve Bayes, becomes
\[ P(C_i|\boldsymbol{x}) = \frac{P(\boldsymbol{x}|C_i) \times P(C)}{P(\boldsymbol{x})}\]
A class $C_i$ is chosen if $P(C_i|x) = \max\limits_{k} \{ P(C_k|x)\}$.

A clear advantage of using Na\"{i}ve Bayes is that, it is fast to train and fast to classify data. This is because it needs to scan the database to compute probabilities and store it in a table during training and use this table to classify future examples. Also Na\"{i}ve Bayes is inherently not sensitive to irrelevant features as likelihood of class is product of probabilities of each single attribute. On the other hand for prior probabilities to be realistic, the sample population needs to be truly representative of actual data. Another major disadvantage is that the classifier assumes features to be independent of each other. However in many cases Na\"{i}ve Bayes classifier performs reasonably well even in cases where features are dependent on each other \cite{John1995, Williams2006}.

\subsection{Decision Trees}
\emph{Decision trees} are hierarchal models, wherein each step is a simple test against a threshold value or nominal attribute against set of all possible values \cite{Kotsiantis2013}. The steps are called as \emph{decision nodes} and test is implemented in form of a function on features $\boldsymbol{x}$ of example, with discrete outcomes represented as branches. These nodes apply tests recursively, as a example flows down the tree until it hits a \emph{leaf node}, which represents output (class in case of classification) \cite{Alpaydin2004}.

\begin{figure}[h]
  \begin{center}
    \captionsetup{justification=centering}
    \includegraphics[scale=0.65]{figures/desctree.png}
    \caption{Example of decision tree}
    \label{fig:desctree}
  \end{center}
\end{figure}

In terms of a computer program, this algorithm devices a set of rules which can be interpreted as nested \texttt{IF-ELSE} structure. A simple decision tree is illustrated in figure~\ref{fig:mlfitting}. \emph{Decision tree learning} algorithms are used to obtain decision trees. ID3, RIPPER, C4.5 are some examples of these algorithms \cite{Mitchell1997}. ID3 \cite{Quinlan1986} is simplest of these algorithms is to choose a feature, which provides the most information about classification data. It then constructs tree using top-down approach. Other advanced algorithms like RIPPER \cite{Cohen1995} build upon same approach and then employ \emph{pruning} to reduce training error.

Decision tree use \enquote{white-box} approach, wherein the internal decision making and structure of tree is visible to user. This also make decision trees easy to visualize and interpret \cite{Kotsiantis2013}. Decision trees also perform feature screening to put less informative features near the leaf nodes, by its construction. Disadvantage of decision trees is that they can create over-complex trees that do not generalize the data well \emph{i.e.} overfitting of data. The problem of learning decision trees is known to be NP-complete hence its worst-case training speed can be slow \cite{Hyafil1976}.

\subsection{Multi-Layer Perceptrons}
\emph{Multi-Layer Perceptrons} (MLP) is a type of artificial neural network models and has been in use since the early 80's. In this model, each feature and outputs are represented as nodes, and feature nodes in each layer are connected to upper layer using weights or \emph{synapses}. Figure~\ref{fig:perceptron} is an example of a simple, single layered perceptron. The inputs $x_1, x_2 \ldots x_k$ are features and $x_0 = +1$ is a \emph{bias element}, used to make model more general by allowing user to fine-tune output by shifting output function.

\begin{figure}[h]
  \begin{center}
    \captionsetup{justification=centering}
    \includegraphics[scale=0.45]{figures/perceptron.png}
    \caption{Single layered perceptron}
    \label{fig:perceptron}
  \end{center}
\end{figure}

The output of perceptron in figure~\ref{fig:perceptron} can be represented mathematically as
\[ y = \sum\limits_{i=1}^k w_ix_i + w_0 \]
During training of perceptron, training algorithm will try to find appropriate connecting weights. Multiple layers of perceptrons can be constructed by implementing hidden layer of nodes between features and output, by doing so one can implement non-linear output functions. The degree of nonlinearity depends on number of hidden layers. Backpropagation algorithm \cite{Rumelhart1985} is one of commonly used algorithm for training MLPs. It works by calculating error correlations at each output and use these to calculate error terms in previous layers and so on. The error terms are then used to adjust weights of individual synapses.

Once trained, MLPs are able to classify data fast \cite{Alpaydin2004}. They can implement higher order polynomial functions and are flexible and powerful due to well researched mathematical background and variety of training algorithms available, which can be selected according to application and amount of data available. A disadvantage is that the network needs to be completely re-trained when new training data is to be added. Selection of features also has a profound impact on performance of MLPs \cite{Kavzoglu2002, El-Khatib2010}.

\subsection{Support Vector Machines}

\section{Criterion To Choose Suitable ML Algorithm}
\label{sec:c3mlselection}

%Die Angabe des schlauen Spruchs auf diesem Wege funtioniert nur,
%wenn keine Änderung des Kapitels mittels den in preambel/chapterheads.tex
%vorgeschlagenen Möglichkeiten durchgeführt wurde.
\chapter{Problem Definition and Feature Selection}
\label{chap:chapter4}
%\vspace{-3cm}
%\vspace{2cm}

There are a number of methods to separate permanent faults from non-recurring faults as explained in the section~\ref{sec:secfc}. However available techniques do not separate faults as permanent, transient and intermittent. This is of particular importance from the point of view of reducing yield loss, by including chips which showed only transient faults. Also, if faults can be categorized into permanent transient or intermittent, then it gives additional information to the designer about the underlying fault mechanism, so that additional optimization of yield can be achieved. 

The yield can be improved by taking the rejected chips, and analyzing them further by running tests again multiple ($TestRuns$) times, and decide on what type of fault caused the failure during the test. These chips that showed a non-critical fault can still be included in the final yield. However an incorrect analysis would result in degradation of product quality. Hence, the classifier should be as accurate as possible.

The criticality of a fault is an abstract concept and it is defined by the application domain of the final product. The classifier to be designed needs to take this fact in consideration and should provide the user with the functionality to find a suitable trade-off between amount of yield and its quality. This is particularly useful when the user is more interested in optimizing for the quality, he would then wish to reject all chips with a slightest possibility of intermittent failure. In this case the classifier should be optimized for classifying intermittent faults more accurately. On the other hand, if the user wants to optimize for yield, classifier should be tuned to classify transient faults more accurately.

The classification approaches explained in section~\ref{sec:secfc} are mostly rule or heuristic based (e.g. the threshold value in $\alpha$-counting techniques). Generally speaking, when an intermittent fault occurs in a system, its activation rate is higher than the transient fault rate \cite{Bondavalli2000}. However, as systems are moving to lower technology nodes, transient faults are also on the rise, as explained in section~\ref{sec:secft}. Hence with traditional techniques, it becomes difficult to separate transients from intermittents using $\alpha$-count, as the fault rates for these two types of faults become close to each other. Hence an elaborate analysis is required to update these rules for every product and technology. Hence, the classifier to be designed should also focus on building a universal model, which can classify faults irrespective of the product and technology.

To summerize, we need an automated and adaptive approach which is independent on product and technology and that can classify faults as intermittent, transient and permanent accurately. 

\section{Machine learning approach for fault classification}

Machine learning has been used in wide variety of classification applications with reasonable accuracy \cite{Pang2002,Nguyen2008,Sebastiani2002, Kotsiantis2007}. As explained in chapter~\ref{chap:chapter3}, machine learning is used when it is not practical to arrive at rules by looking at the data. Machine learning algorithms can be implemented as \enquote{black-box} approach for classification and all user needs to do is adjust a few parameters, depending on machine learning algorithm used. Even parameter searching can be automated and user can fine-tune them for further improvement in accuracy \cite{Hsu2003, Castillo2000}. This makes machine learning a practical and automated approach when large amount of data is available.

Once a feature set is fixed and the required parameters are decided, machine learning algorithms analyze the data to set up a classification model. When a technology node is updated one might have to change the database and train the algorithm again, but the training algorithm takes care of feature space and classification rules, making machine learning approaches adaptive.

Figure~\ref{fig:mlsteps} explains basic steps for classification using machine learning methods.

\begin{figure}[h]
  \begin{center}
    \captionsetup{justification=centering}
    \includegraphics[scale=0.65]{figures/mlsteps.png}
    \caption{Classification of VLSI chips using machine learning}
    \label{fig:mlsteps}
  \end{center}
\end{figure}

First step in designing a machine learning system for classification is to decide on features, which can be efficiently classify the data. Selection of proper features has the most impact in accuracy of any classifier \cite{Michie1994}. This, along with design of algorithms to extract features from input data is explained in section~\ref{sec:secfs} of this chapter. Next step is to generate sample population and decide on which machine learning algorithm is to be used, covered in next chapter.

\section{Feature Selection}
\label{sec:secfs}
To begin with feature selection, it is important to take a look at all the behavioral characteristics of different types of faults, as covered in chapter~\ref{chap:chapter2}. Table~\ref{tab:charfaults} summarizes all important characteristics to be considered for selection of features. Rest of the section describes features that were selected and algorithms for extraction of those features.

{%
\newcommand{\mc}[3]{\multicolumn{#1}{#2}{#3}}
\begin{table}[H]
 \begin{center}
  \captionsetup{justification=centering}
  \begin{tabular}{lp{4cm}p{4cm}p{4cm}}
	\hline
    \mc{1}{c}{\textbf{Characteristic}} & \mc{1}{c}{\textbf{Permanent faults}} & \mc{1}{c}{\textbf{Intermittent faults}} & \mc{1}{c}{\textbf{Transient faults}}\\ \hline
    Affected outputs & Affects same set of output pins & Affects same set of output pins & Affects any of primary outputs\\
    Reproducibility & Reproducible for same test vector & Sometimes reproducible for same test vector depending upon the error activation rate & Not reproducible\\
    Location on chip & Fixed to a fault location & Localized to a fault location & Can affect any location on chip\\
    Fault behavior & Deterministic & Non-deterministic & Non-deterministic \\ \hline
  \end{tabular}
  \caption{Characteristics of faults in VLSI systems}
  \label{tab:charfaults}
 \end{center}
\end{table}
}%

\subsection{Reproducibility of fault}
Reproducibility of a fault pattern during multiple test runs is defined as maximum number as maximum number of occurrence of same faulty output pattern for a fixed input pattern, and it is denoted by $\epsilon$. 

\begin{figure}[h]
  \begin{center}
    \captionsetup{justification=centering}
    \includegraphics[scale=0.65]{figures/epsilon.png}
    \caption{Expected behavior of $\epsilon$ for different faults}
    \label{fig:epsilon}
  \end{center}
\end{figure}

Algorithm~\ref{alg:epsilon} explains extraction of $\epsilon$. It takes the expected and actual output patterns as inputs. It then checks if any of output patterns was faulty and it calculates maximum occurrences of every faulty pattern for given input pattern and stores in into array. Final value of $\epsilon$ is maximum value of in this array.

\begin{algorithm}[H]
  \caption{Algorithm to evaluate $\epsilon$}
  \label{alg:epsilon}
  \begin{algorithmic}
 \Procedure{ComputeEpsilon}{Expected output pattern array (EX), Observed output pattern array for all test runs (OP)}
 \State $\epsilon$[size(EX)] $\leftarrow$ 0\;
 \State Index $\leftarrow$ 0\;
 \While{Index $<$ size(EX)}
  \If{EX[Index] $\neq$ any pattern of OP[Index][]}
   \State $\epsilon$[Index] $\leftarrow$ max(\Call{SimilarFaultyPatterns}{OP[Index][]})\;
  \Else
   \State $\epsilon$[Index] $\leftarrow$ 0\;
  \EndIf
  \State Index++\;
 \EndWhile
 \State$\epsilon$ $\leftarrow$ max($\epsilon$[])\;
 \State \Return $\epsilon$\;
 \EndProcedure
 \end{algorithmic}
\end{algorithm}

Figure~\ref{fig:epsilon} shows expected behavior of $\epsilon$ for different fault types. Permanent faults are repeatable and value of $\epsilon$ is expected to be equal to the number of test runs for these type of faults. Intermittent faults occur at higher rate than that of transients for a fixed input pattern, hence they are also expected to have somewhat higher value than transient faults. Figure~\ref{fig:epsilonp45k} shows actual values of $\epsilon$ for a simple circuit (p45k).

\begin{figure}[h]
  \begin{center}
    \captionsetup{justification=centering}
    \includegraphics[scale=0.35]{figures/epsilonp45k.png}
    \caption{Plot of $\epsilon$ for p45k}
    \label{fig:epsilonp45k}
  \end{center}
\end{figure}

\subsection{Resemblance of erroneous output patterns}
Resemblance of erroneous output patterns is defined in terms of hamming distance between a set of erroneous output patterns obtained during multiple test runs, for the same input test pattern in a test set. \emph{Hamming Distance} of a set is evaluated as maximum of number of positions in which output patterns differ, pairwise. It is denoted using notation $\delta_H$, subscript $H$ stands for \enquote{horizontal}, denoting the orientation of calculation of the hamming distance. If all output patterns for an input test pattern are correct then the hamming distance and hence the value of $\delta_H$ would be zero.

\begin{figure}[h]
  \begin{center}
    \captionsetup{justification=centering}
    \includegraphics[scale=0.65]{figures/deltah.png}
    \caption{Expected behavior of $\delta_H$ for different faults}
    \label{fig:deltah}
  \end{center}
\end{figure}

Algorithm~\ref{alg:deltah} explains extraction of $\delta_H$. It takes the expected and actual output patterns as inputs. It then checks if any of output pattern is faulty, to save some computational efforts. If  any of output pattern is faulty, it calculates value of $\delta_H$ and stores it in array against corresponding index of input pattern. Final value of $\delta_H$ is maximum value of $\delta_H$ in this array. 

\begin{algorithm}[H]
  \caption{Algorithm to evaluate $\delta_H$}
  \label{alg:deltah}
  \begin{algorithmic}
 \Procedure{ComputeDeltaH}{Expected output pattern array (EX),Observed output pattern array for all test runs (OP)}
 \State $\delta_H$[size(EX)] $\leftarrow$ 0\;
 \State Index $\leftarrow$ 0\;
 \While{Index $<$ size(EX)}
  \If{EX[Index] $\neq$ any pattern of OP[Index][]}
   \State $\delta_H$[Index] $\leftarrow$ \Call{HammingDistance}{OP[Index][]}\;
  \Else
   \State $\delta_H$[Index] $\leftarrow$ 0\;
  \EndIf
  \State Index++\;
 \EndWhile
 \State$\delta_H$ $\leftarrow$ max($\delta_H$[])\;
 \State \Return $\delta_H$\;
 \EndProcedure
 \end{algorithmic}
\end{algorithm}

Figure~\ref{fig:deltah} shows expected behavior of $\delta_H$ for different fault types. Permanent faults repeat with same faulty output pattern and value of $\delta_H$ is expected to be zero. Intermittent faults, even though fail with same faulty output, are not repeatable and hence are expected to have a $\delta_H$ value other than zero. Transient fail randomly at random output locations and hence are expected to have higher $\delta_H$ value. Figure~\ref{fig:deltahp45k} shows actual values of $\delta_H$ for a simple circuit (p45k).

\begin{figure}[h]
  \begin{center}
    \captionsetup{justification=centering}
    \includegraphics[scale=0.35]{figures/deltahp45k.png}
    \caption{Plot of $\delta_H$ for p45k}
    \label{fig:deltahp45k}
  \end{center}
\end{figure}

\subsection{Resemblance of erroneous primary outputs}
Resemblance of erroneous primary outputs is defined as hamming distance between primary output locations, that showed a faulty behavior at least once for a respective test run. This quantity is denoted by $\delta_V$, subscript $V$ denoting \emph{vertical} collapsing of all faulty primary outputs for a test run, before evaluating hamming distance.

Algorithm~\ref{alg:deltav} explains extraction of $\delta_V$. It takes expected output pattern and actual output patterns as inputs. It then marks the pins which showed faulty behavior at least once in a single test run of complete pattern set, as dirty. Hence number of elements in array of $\delta_V$ equals number of test runs. The final value of $\delta_V$ is hamming distance of $\delta_V$ array.

\begin{algorithm}[H]
  \caption{Algorithm to evaluate $\delta_V$}
  \label{alg:deltav}
  \begin{algorithmic}
 \Procedure{ComputeDeltaV}{Expected output pattern array (EX),Observed output pattern array for all test runs (OP)}
 \State $\delta_V$[TotalRuns] $\leftarrow$ 0\;
 \State CurrentRun $\leftarrow$ 0\;
 \While{CurrentRun $<$ TotalRuns}
  \State Index $\leftarrow$ 0\;
  \While{Index $<$ size(EX)}
	\State ExpectedOutput $\leftarrow$ EX[Index]\;
	\State ActualOutput $\leftarrow$ OP[Index][CurrentRun]\;
   \State Iterator $\leftarrow$ 0\;
	\While{Iterator $<$ length(ExpectedOutput)}
		\If{ExpectedOutput.CharAt(Iterator) $\neq$ ActualOutput.CharAt(Iterator)}
		 \State $\delta_V$[CurrentRun].CharAt(Iterator) $\leftarrow$ 1\;
		\Else
		 \State $\delta_V$[CurrentRun].CharAt(Iterator) $\leftarrow$ 0\;
		\EndIf
		\State Iterator++\;
		\EndWhile
	\State Index++\;
  \EndWhile
 \State CurrentRun++\;
 \EndWhile
 \State $\delta_V$ $\leftarrow$ \Call{HammingDistance}{$\delta_V$[]}\;
 \State \Return $\delta_V$\;
 \EndProcedure
 \end{algorithmic}
\end{algorithm}

It is expected that value of $\delta_V$ is low in case of permanent and intermittent faults, as these faults manifest into failures at fixed set of output pins. In contrast, transient faults do not have a fixed set of outputs that it affects,resulting in high expected value of $\delta_V$.  Figure~\ref{fig:deltavp45k} shows actual values of $\delta_V$ for a simple circuit (p45k).

\begin{figure}[h]
  \begin{center}
    \captionsetup{justification=centering}
    \includegraphics[scale=0.35]{figures/deltavp45k.png}
    \caption{Plot of $\delta_V$ for p45k}
    \label{fig:deltavp45k}
  \end{center}
\end{figure}

\subsection{Diagnostic features}
Diagnostic features from section~\ref{sec:secfd} are also considered as features for learning algorithm. Diagnostic data also provides information about fault present in circuit. A short summery of fault models and observed behavior for diagnostic parameters is summarized in table~\ref{tab:diagparam}. This table is taken from original work by Holst \emph{et.al.} \cite{Holst2009}.

\begin{table}[h]
\captionsetup{justification=centering}

\begin{tabular}{p{2cm}p{2.5cm}p{2.5cm}p{2.5cm}p{2.5cm}}
\hline
Fault type   & \multicolumn{1}{c}{$\sigma$}                              & \multicolumn{1}{c}{$\iota$}                           & \multicolumn{1}{c}{$\tau$}                            & \multicolumn{1}{c}{$\gamma$}                                                   \\ \hline
Permanent    & \textgreater0, same sigma values in all of the test runs. & \textgreater0, only with transient noise, otherwise 0 & \textgreater0, only with transient noise, otherwise 0 & \textgreater0, only for delay fault model, same values in all of the test runs \\
Intermittent & \textgreater0, low as compared to permanent faults        & \textgreater0, higher as compared to permanent faults & \textgreater0, only with transient noise, otherwise 0 & 0                                                                              \\
Transient    & \textgreater0, changes every test run                     & \textgreater0, changes every test run                 & \textgreater0, changes every test run                 & \textgreater0, changes every test run                                         \\
\hline
\end{tabular}

\caption{Expected evidence values for various fault types}
\label{tab:diagparam}
\end{table}


In this work it is assumed that at most only a single intermittent or permanent fault is active in the circuit with or without some transient noise (Reference to assumptions section in chap 5). Most of permanent faults can be modeled as single stuck-at, single conditional stuck-at or a delay faults. Transient faults can be modeled as conditional stuck-at faults, at multiple fault sites as the do not have a fixed location on chip and some deterministic probability function can be used as trigging condition. Similarly intermittent faults can be modeled as single conditional stuck-at faults. Hence combination of values of parameters can be used to deduce which type of fault might exist on the chip.

Furthermore, the standard deviation of evidence parameters is expected to be around zero for permanent faults, as once they are detected then they always be detected using same test pattern set. Hence as standard deviations of evidence based features convey some information about fault class, they also considered as features for learning. Figure~\ref{fig:sdevidance} shows actual values of standard deviation of evidence based features for a simple circuit (p45k).

\begin{figure}
        \centering
			\captionsetup{justification=centering}
        \begin{subfigure}[h]{0.45\linewidth}
                \includegraphics[scale=0.32]{figures/sdsigmap45k.png}
                \caption{Sigma ($\sigma$)}
        \end{subfigure}
        \begin{subfigure}[h]{0.45\linewidth}
                \includegraphics[scale=0.32]{figures/sdiotap45k.png}
                \caption{Iota ($\iota$)}
        \end{subfigure}
			\begin{subfigure}[h]{0.45\linewidth}
                \includegraphics[scale=0.32]{figures/sdtaup45k.png}
                \caption{Tau ($\tau$)}
        \end{subfigure}
        \caption{Standard deviations for evidence-based parameters for p45k}
        \label{fig:sdevidance} 
\end{figure}

Table~\ref{tab:features} summarizes the selected features.
{%
\newcommand{\mc}[3]{\multicolumn{#1}{#2}{#3}}

\captionsetup{justification=centering}
\begin{table}[h]
\begin{tabular}{lll}
\hline
\mc{1}{c}{\textbf{Category}}        & \mc{1}{c}{\textbf{Feature}}       & \mc{1}{c}{\textbf{Symbol}}\\
\hline
\multirow{3}{*}{Non-evidence based features} & Reproducible of fault                    & $\sigma$\\
                                             & Resemblance of erroneous output patterns  & $\delta_H$\\
                                             & Resemblance of erroneous primary outputs   & $\delta_V$\\
\hline
\multirow{8}{*}{Evidence based features}     & Maximum $\sigma$ among all test runs          & $\sigma$\\
                                             & $\iota$ corresponding to maximum $\sigma$  & $\iota$                             \\
                                             & $\tau$ corresponding to maximum $\sigma$   & $\tau$ \\
                                             & $\gamma$ corresponding to maximum $\sigma$ & $\gamma$ \\
                                             & Standard deviation of $\sigma$             & SD($\sigma$)\\
                                             & Standard deviation of $\iota$              & SD($\iota$)\\
                                             & Standard deviation of $\tau$               & SD($\tau$)\\
                                             & Standard deviation of $\gamma$             & SD($\gamma$)\\
\hline
\end{tabular}
\caption{Summary of selected features}
\label{tab:features}
\end{table}
}%

\section{Selection of machine learning algorithm for fault classification}
\label{sec:selml}

There are variety of machine learning algorithms available, and some of the important ones were surveyed in chapter~\ref{chap:chapter3}. The accuracy of machine learning classifier depends mainly on the complexity of the feature space and actual training data at hand. Now that the feature set known, selection of classifier is done with respect to the following factors:

\begin{description}
  \item[Feature set] The feature set is not statistically independent, an important consideration as it violates the prerequisite for Bayes classifier. It can be seen from plots presented in section~\ref{sec:secfs}, that the feature space has high variance and it is not linearly separable, thus it is not practical to come up with rules to classify faults and hence, the performance of decision trees can be expected to be on the lower side. Also, as the feature space is highly complex, polynomial functions in MLPs might not be sufficient to create an acceptable hypothesis, can take long time to train using higher order polynomials and can result overfitting the training data. SVMs, on the other hand can handle kernel functions and can be expected to create a complex hypothesis, as required in our case.

  \item[Sample population] Size of our training data set is limited, as it needs to be generated on a simulator(refer section~\ref{sec:gsp}). Neural networks and decision trees are known to work well with large training sets \cite{DeFries2000,Tanwani2009}. On the other hand, SVMs are shown to be effective with limited size of data sets\cite{Koggalage2004}. The sample population that we have used is not created according to the probabilities of fault occurrence in real world data. It is also not possible to do so otherwise because of two main reasons, first, it is hard to get hands on actual production numbers for evaluation of results, as those numbers a closely guarded company secrets. And second, in actual application scenario, it is hard to predict probabilities of fault occurrence. This makes impossible to set set up prior probabilities required for Bayes classifier.
\end{description}

With this, it becomes clear that SVM can be used as classifier of choice as:
\begin{itemize}
  \item SVMs can work well with small and medium size data sets, with relatively high accuracy \cite{Koggalage2004, Matlab2014}.
  \item They can handle n-dimensional feature spaces.
  \item By construction, they can handle complex of feature spaces with use of kernel functions.
  \item Once trained, they are fast to classify data .
\end{itemize}

%Die Angabe des schlauen Spruchs auf diesem Wege funtioniert nur,
%wenn keine Änderung des Kapitels mittels den in preambel/chapterheads.tex
%vorgeschlagenen Möglichkeiten durchgeführt wurde.
\chapter{Experimental Setup}
\label{chap:chapter5}
%\vspace{-3cm}
%\vspace{2cm}
With the feature selection and their extraction explained in earlier chapter, we ca now move to the methodology for experiments. This chapter focuses on three topics - first, being the assumptions for the test setup and the configuration of the sample population as well as the method used to generate the sample population, explained in the section~\ref{sec:gsp}. Second, criteria for selection of the machine learning algorithm among those already surveyed in the chapter~\ref{chap:chapter3} is described in the section~\ref{sec:selml}. Third part, in the section~\ref{sec:mltools} briefly describes tools surveyed for using the selected machine learning algorithm.
\section{Generation of sample population}
\label{sec:gsp}
Sample population is required to train and cross-validate the machine learning based classifier (hereon referred simply as \emph{classifier}). A separate set of data is used to test the accuracy of the classifier.
\subsection{Assumptions}
\label{sec:gsp:assumptions}
Following are the assumptions while generating the sample population and test data:
\begin{enumerate}
  \item Sample population is generated per circuit type \emph{i.e.} all examples in sample population for a circuit type are generated by simulating a single netlist with different fault instances. 
  \item It is assumed that the chips to be classified have displayed faulty behavior and hence were rejected in earlier test.
  \item The chips under consideration are either:
		  \begin{enumerate}
    		\item Healthy chips with transient noise or
    		\item Affected by a single permanent or intermittent fault, with or without transient noise.
 		 \end{enumerate}
	\item Permanent faults are modeled using stuck-at, wired or delay faults, Intemittents are modeled using high frequency power droop, and transient faults are modeled as conditional stuck-at faults at random locations, triggered using deterministic fault rate.
	\item Number of simulation runs to extract features is fixed at 4. This value is set experimentally. 
\end{enumerate}

\subsection{Configuration}
For the purpose of running experiments uniformly on different circuit types, a sample population and a test set for each of cicuit types is created with following configuration:
\begin{enumerate}
  \item A sample population consist of 2500 ($\pm$ 75) of labeled examples. The tolerance of $\pm$ 75 is set as the permanent or intermittent fault instances which did not show any faulty behavior at POs at all were removed.
  \item The sample population is equally divided into following five fault categories:
		\begin{enumerate}
    		\item Permanent faults with label \texttt{P}.
    		\item Permanent faults along with transient noise (fault rate = 0.001) with label \texttt{P}.
			\item Intermittent faults (fault rate = 0.1, 0.01 0.001) with label \texttt{I}.
    		\item Intermittent faults (fault rate = 0.1, 0.01, 0.001) along with transient noise (fault rate = 0.001) with label \texttt{I}.
			\item Transient faults (fault rate = 0.01, 0.001, 0.0001) with label \texttt{T}.
 		 \end{enumerate}
  \item One-fourth of total population is separated out as cross-validation set, rest is used as training data.
  \item Test data has 250 ($\pm$ 15) labeled examples, with same configuration as that for sample population.
\end{enumerate}

\subsection{Implementation}

Sample population is generated as shown in figure~\ref{fig:sampopl} using a simulation framework called Adaptive Diagnosis of Arbitrary Manifold Artifacts (ADAMA). ADAMA can be used for logic simulation with error injection. Logical representation of a circuit at gate level and test pattern set are the required inputs for simulation using ADAMA. A fault description can be provided optionally to inject a fault and analyze its behavior. ADAMA supports all of the fault models that we have considered under our assumptions in section~\ref{sec:gsp:assumptions}.

\begin{figure}[h]
  \begin{center}
    \captionsetup{justification=centering}
    \includegraphics[scale=0.45]{figures/sampopl.png}
    \caption{Generation of sample population using \texttt{spgen}}
    \label{fig:sampopl}
  \end{center}
\end{figure}

For experiments, ADAMA framework has been extended by adding a task to generate sample population, called \texttt{spgen}.\texttt{spgen} takes fault description, circuit description and test pattern set as inputs along with number of simulations runs (\texttt{simruns}) to be performed for feature extraction. It then runs simulation and diagnosis \texttt{simruns} times and passes it on to the object of class \texttt{MLFeatureExtractor}, which encompasses all of the procedures for feature extraction described in section~\ref{sec:secfs} of previous chapter. Once complete simulation is over, it writes the features and its label in a CSV file.

Shell scripts are used to invoke instances of ADAMA and inject fault instances of permanent, transient and intermittent faults. The process to do the same is described below:

\begin{description}
  \item[Permanent Faults] A task in ADAMA called \texttt{fsample} is used to generate fault descriptions for permanent faults. Script first runs task \texttt{fsample} and puts all fault descriptions in a file, and then parses it line-by-line to invoke instance of ADAMA with task \texttt{spgen}. It then uses same fault descriptions and adds a transient fault instance to generate transient noise and runs the complete process again, while logging seed values used to generate transients. After generation of features is done, it scans through the feature file of permanent faults without transient noise and scans for fault instances which did not result in failure at POs and removes corresponding features, fault descriptions and corresponding items in feature file for permanent faults with transient noise.

  \item[Intermittent faults] Script first generates intermittent fault descriptions with random seed values for location and fault activation and store them into a file. It then invokes \texttt{spgen} using ADAMA and simulates these fault instances first without and then with transient noise, while logging all seed values and corresponding fault rates. The fault descriptions and and corresponding examples in feature files of intermittent faults and intermittent faults with transient noise are removed, where intermittent fault as not active at least for one simulation round.

  \item[Transient faults] Script takes circuit description and passes them to \texttt{spgen} in ADAMA and logs seed values and corresponding fault rates used for generation of transient fault instances.
\end{description}

\section{Selection of machine learning algorithm for fault classification}
\label{sec:selml}

With feature set finalized and sample population generated, selection of classifier is done using the following factors:

\begin{description}
  \item[Feature set] The feature set is not statistically independent, an important consideration as it violates the prerequisite for Bayes classifier. It can be seen from plots presented in section~\ref{sec:secfs}, that the feature space has high variance and it is not linearly separable, thus it is not practical to come up with rules to classify faults and hence, the performance of decision trees can be expected to be on the lower side. With the same logic, as the feature space is highly complex, polynomial functions in MLPs might not be sufficient to create an acceptable hypothesis, can taking long time to train using higher order polynomials, and can result overfitting the training data. SVMs, on the other hand can handle kernel functions, and can be expected to create a complex hypothesis, as required in our case.

  \item[Sample population] Size of our training dataset is limited. Neural networks and decision trees are known to work well with large training sets \cite{DeFries2000,Tanwani2009}. On the other hand, SVMs are shown to be effective with limited size of \cite{Koggalage2004}. The sample population that we have used is not created according to the probabilities of fault occurrence in real world data. It is also not possible to do so otherwise because of two main reasons, first, it is hard to get hands on actual production numbers for evaluation of results, as those numbers a closely guarded company secrets. And second, in actual application scenario, it is hard to probabilities of fault occurrence. This makes impossible to set set up prior probabilities required for Bayes classifier.
  \item[Speed and accuracy] \cite{Matlab2014} \ldots
\end{description}

\subsection{SVM for fault classification}
\subsection{SVM classifier libraries}
\section{Tools for machine learning using SVM}
\label{sec:mltools}
\subsection{LIBSVM}
\subsection{Other tools}
\subsection{Interface}
\subsection{Available Kernels}
\subsection{Training and Parameter selcetion}

%Die Angabe des schlauen Spruchs auf diesem Wege funtioniert nur,
%wenn keine Änderung des Kapitels mittels den in preambel/chapterheads.tex
%vorgeschlagenen Möglichkeiten durchgeführt wurde.
\chapter{Evaluation of results}
\label{chap:chapter6}
%\vspace{-3cm}
%\vspace{2cm}
This chapter presents results of classification using the experimental setup described in the chapter~\ref{chap:chapter6}. The term \emph{Classification accuracy} of a class used in this chapter is defined as,

\[ \begin{array}{ll} accuracy(r_i) = \frac{\mbox{Samples correctly predicted as $r_i$}}
								{\mbox{Total samples with label $r_i$}}
							\times 100 & 
							\forall \,r_i \in \{\mbox{\texttt{P,I,T}}\} 
	\end{array}\]

Where, $r$ is the label of test sample $x=\{\boldsymbol{x},r\}$.

In the presented evaluation, the cross-validation accuracy is also noted for each set of experiments, as this is the value of accuracy during the self-validation of much larger training set.

In the first part analyzes classification efficiency after the grid search for optimal values of $C$ and $\gamma$, but does not consider any other optimization, a detailed account of results are presented for each kernel type available in \texttt{LIBSVM}. The second part improvises the first, and results are noted after optimization of so-called \emph{class-weights}. The third part explores two possibilities - that of using one of the using a single prediction model for predicting all types of circuits, and the second, to extrapolate a known model to predict new circuit types. The last part tries to evaluate the impact on yield and the quality of the final product after application of the suggested method.

\section{Classification of permanent faults}
\label{sec:wp}
Consider a special case, where $\epsilon = Test Runs$. In our classification problem this can mean,
\begin{enumerate}
  \item It is a permanent fault.
  \item A rare case of highly repetitive intermittent fault, and can be assumed to be \enquote{critical}.
  \item An extremely rare case of a transient fault, as it happened at the same location and for same input test pattern, in all of the test runs.
\end{enumerate}

In the other direction, for permanent faults $\epsilon = Test Runs$ holds true, except a rare possibility that the transient noise affects all of the test patterns, which have resulted in a failure at POs, in at least one of the test runs. This is also an extremely rare possibility. Hence we assume that, a fault is permanent if and only if $\epsilon = Test Runs$.

If we go back to the section~\ref{sec:secfs} and observe the plot for $\epsilon$ in the figure~\ref{fig:epsilonp45k}, we can observe our assumption to be fairly accurate. Hence if we do a slight modification in our experimental setup and already remove faults where $\epsilon = Test Runs$, and classify them as permanent faults, we can achieve 100\% permanent fault classification, and we would be left with a binary classification problem between transient and intermittent faults. 

\section{Classification without class-weight optimization}
Class-weights are simple multipliers to parameter $C$ of the classifier of respective class. The first set of experiments consist of evaluation of accuracy levels without assigning class weights \emph{i.e.} class-weights are assumed to be \{1,1,1\} for permanent, intermittent and transient faults respectively. The experiments are carried out for variety of circuits ranging from a simple p45k to a highly complex p295k. The experiments are repeated for each type of kernel provided by \texttt{LIBSVM}. 

This set of experiments consist of two rounds each for each kernel, first one includes permanent faults in the sample population, the other one without considering them. A round with permanents included, is carried out only to evaluate the relative change in accuracy levels of classification.

\subsection{Linear Kernel}
Accuracy of SVM using a linear kernel is summarized in table~\ref{tab:linwp} with considering permanent faults in the sample population, and the same without considering permanents is summarized in the table~\ref{tab:linwop}.
\begin{table}[h]

	\captionsetup{justification=centering}
\begin{tabular}{ccccccc}
\hline
\multicolumn{1}{c}{\multirow{3}{*}{Circuit}} & \multicolumn{6}{c}{Accuracy (\%)}\\ \cline{2-7} 
\multicolumn{1}{c}{}                         & \multicolumn{1}{c}{\multirow{2}{*}{Cross-validation}} & \multicolumn{2}{c}{Permanent} & \multicolumn{2}{c}{Intermittent} & \multicolumn{1}{c}{\multirow{2}{*}{Transient}} \\ \cline{3-6}
                                             &                                                       & w/o noise     & with noise    & w/o noise      & with noise      &                                                \\ \hline
p45k                                         & 90.14                                                 & 87.50         & 87.50         & 77.50          & 64.58           & 93.90                                          \\
p100k                                        & 87.38                                                 & 100           & 100           & 80.95          & 57.14           & 95.91                                          \\
p141k                                        & 83.89                                                 & 98.21         & 98.21         & 61.90          & 57.14           & 100                                            \\
p267k                                        & 85.12                                                 & 100           & 100           & 55.00          & 52.50           & 100                                            \\
p279k                                        & 89.52                                                 & 98.30         & 96.61         & 78.94          & 52.50           & 89.79                                          \\
p295k                                        & 89.84                                                 & 100           & 100           & 81.39          & 65.11           & 100   \\
\hline                                        
\end{tabular}
\caption {Classification accuracy for linear kernel}
\label{tab:linwp}
\end{table}

In both cases, with and without permanent faults, it can be observed that, permanent faults are being relatively accurately classified and background transient noise seems to have almost no effect on the classification accuracy of permanent faults. Intermittent faults in both cases have relatively low accuracy levels and tend to deteriorate severely in presence of transient noise. 


\begin{table}[h]
\captionsetup{justification=centering}
\begin{tabular}{ccccc}
\hline
\multirow{3}{*}{Circuit} & \multicolumn{4}{c}{Accuracy (\%)}\\ \cline{2-5} 
                         & \multirow{2}{*}{Cross-validation} & \multicolumn{2}{c}{Intermittent} & \multirow{2}{*}{Transient} \\ \cline{3-4}
                         &                                   & w/o noise      & with noise      &                            \\ \hline
p45k                     & 87.81                             & 72.91          & 70.08           & 97.95                      \\
p100k                    & 81.43                             & 69.05          & 59.52           & 97.96                      \\
p141k                    & 84.13                             & 78.57          & 76.19           & 100                        \\
p267k                    & 77.09                             & 80.00          & 55.00           & 100                        \\
p279k                    & 84.65                             & 78.94          & 52.63           & 91.83                      \\
p295k                    & 89.17                             & 90.69          & 81.39           & 100           \\
\hline            
\end{tabular}
\caption {Classification accuracy for linear kernel, without permanent faults}
\label{tab:linwop}
\end{table}

After removing permanents from the sample population, classification accuracies are observed to improve, with an exception of p100k. However, accuracy levels are observed to increase for intermittents with noise, even in the case of p100k. Classification accuracy of transient faults has also increased. Cross-validation accuracy levels have decreased, as a large chuck of correctly classified data in earlier case was permanent faults. Comparatively lower accuracy levels for intermittent faults and almost 100\% classification rate for transients suggests that, linear kernel is more biased towards transient faults.

\subsection{Polynomial Kernel}
\begin{table}[h]

	\captionsetup{justification=centering}
\begin{tabular}{ccccccc}
\hline
\multirow{3}{*}{Circuit} & \multicolumn{6}{c}{Accuracy (\%)}\\ \cline{2-7} 
                         & \multirow{2}{*}{Cross-validation} & \multicolumn{2}{c}{Permanent} & \multicolumn{2}{c}{Intermittent} & \multirow{2}{*}{Transient} \\ \cline{3-6}
                         &                                   & w/o noise     & with noise    & w/o noise      & with noise      &                            \\ \hline
p45k                     & 90.92                             & 87.50         & 89.58         & 68.75          & 70.83           & 83.67                      \\
p100k                    & 89.70                             & 100           & 97.82         & 69.04          & 66.66           & 93.87                      \\
p141k                    & 87.94                             & 100           & 100           & 73.80          & 71.42           & 95.00                      \\
p267k                    & 87.34                             & 100           & 96.55         & 82.50          & 60.00           & 85.71                      \\
p279k                    & 91.42                             & 98.30         & 96.55         & 81.57          & 73.68           & 91.83                      \\
p295k                    & 90.78                             & 100           & 98.21         & 76.09          & 72.09           & 95.91      \\
\hline                                                     
\end{tabular}
\caption {Classification accuracy for polynomial kernel}
\label{tab:polywp}
\end{table}

Accuracy of SVM using a polynomial kernel is summarized in table~\ref{tab:polywp} with considering permanent faults in the sample population, and the same without considering permanents is summarized in table~\ref{tab:polywop}.

Accuracy of classification of permanent faults, without noise in case of the polynomial kernel is observed to be even higher than that for linear kernel. For permanents with noise, classification rates are almost similar. Intermittent faults with permanents in first round, without noise are deteriorating in some cases, while improving significantly in others. With noise, however, accuracy of intermittent fault classification is improved. Classification accuracy of transient faults is decreased as compared to the linear kernel.

\begin{table}[h]
\captionsetup{justification=centering}
\begin{tabular}{ccccc}
\hline
\multirow{3}{*}{Circuit} & \multicolumn{4}{c}{Accuracy (\%)}\\ \cline{2-5} 
                         & \multirow{2}{*}{Cross-validation} & \multicolumn{2}{c}{Intermittent} & \multirow{2}{*}{Transient} \\ \cline{3-4}
                         &                                   & w/o noise      & with noise      &                            \\ \hline
p45k                     & 87.81                             & 81.25          & 72.91           & 87.76                      \\
p100k                    & 85.00                             & 78.57          & 66.66           & 89.79                      \\
p141k                    & 85.57                             & 80.95          & 76.19           & 100                        \\
p267k                    & 81.38                             & 82.50          & 72.50           & 93.87                      \\
p279k                    & 85.57                             & 78.94          & 73.68           & 93.87                      \\
p295k                    & 92.28                             & 93.02          & 86.04           & 93.87                      \\
\hline
\end{tabular}
\caption {Classification accuracy for polynomial kernel, without permanent faults}
\label{tab:polywop}
\end{table}

In the second round, after removing permanents, a huge improvement in accuracy of intermittent faults is observed as compared to the first round. The results are also better than those for the linear kernel.  A marginal improvement in transient fault classification accuracy is also observed as compared to the first round. However, overall figures for transients are lower than those for linear kernel. Observations suggest that, the polynomial kernel is slightly biased towards intermittents.


\subsection{RBF Kernel}

Accuracy of SVM using a Radial Basis Function (RBF) kernel is summarized in table~\ref{tab:rbfwp} with considering permanent faults in the sample population, and the same without considering permanents is summarized in table~\ref{tab:rbfwop}.

\begin{table}[h]
\captionsetup{justification=centering}
\begin{tabular}{ccccccc}
\hline
\multirow{3}{*}{Circuit} & \multicolumn{6}{c}{Accuracy (\%)}\\ \cline{2-7} 
                         & \multirow{2}{*}{Cross-validation} & \multicolumn{2}{c}{Permanent} & \multicolumn{2}{c}{Intermittent} & \multirow{2}{*}{Transient} \\ \cline{3-6}
                         &                                   & w/o noise     & with noise    & w/o noise      & with noise      &                            \\ \hline
p45k                     & 91.21                             & 100           & 100           & 66.66          & 64.58           & 91.83                      \\
p100k                    & 87.36                             & 100           & 100           & 64.28          & 61.94           & 93.87                      \\
p141k                    & 88.63                             & 100           & 100           & 73.80          & 71.42           & 93.87                      \\
p267k                    & 87.27                             & 100           & 98.27         & 80.00          & 72.50           & 93.87                      \\
p279k                    & 89.52                             & 98.30         & 96.55         & 78.92          & 71.05           & 93.87                      \\
p295k                    & 91.00                             & 100           & 100           & 79.06          & 74.41           & 95.91                     \\
\hline                                                     
\end{tabular}
\caption {Classification accuracy for RBF kernel}
\label{tab:rbfwp}
\end{table}
Classification rates for permanent faults, with and without noise, are higher for RBF kernel as compared to others. Like in other cases, injection of the transient noise has affected the accuracy of intermittent fault classification, significantly. Accuracy figures for both, intermittent faults with and without noise, are lower as compared to previous two kernels. Transient fault classification accuracy shows a marginal improvement over polynomial kernel, but is still lower than the linear kernel.

Intermittent fault classification accuracy is increased for simpler circuits, but shows no change for more complex circuits. However, for intermittent faults injected with noise, accuracy is improved as compared to the first round, but is still lower than other kernel types. A slight increase in transient fault classification accuracy is also observed. 

A notable observation in case of RBF kernel is its comparatively higher cross-validation accuracy figures and lower test set classification accuracy, an indication that the kernel may be overfitting.

\begin{table}[h]
	\captionsetup{justification=centering}
\begin{tabular}{ccccc}
\hline
\multirow{3}{*}{Circuit} & \multicolumn{4}{c}{Accuracy (\%)}                                                                 \\ \cline{2-5} 
                         & \multirow{2}{*}{Cross-validation} & \multicolumn{2}{c}{Intermittent} & \multirow{2}{*}{Transient} \\ \cline{3-4}
                         &                                   & w/o noise      & with noise      &                            \\ \hline
p45k                     & 87.56                             & 77.08          & 70.83           & 91.83                      \\
p100k                    & 83.33                             & 78.57          & 64.28           & 89.79                      \\
p141k                    & 86.53                             & 83.33          & 80.95           & 100                        \\
p267k                    & 80.19                             & 82.50          & 70.00           & 92.34                      \\
p279k                    & 85.64                             & 78.92          & 76.31           & 95.91                      \\
p295k                    & 85.64                             & 79.06          & 76.31           & 95.91  						  \\
\hline
\end{tabular}
\caption {Classification accuracy for RBF kernel, without permanent faults}
\label{tab:rbfwop}
\end{table}

\subsection{Sigmoid Kernel}

Accuracy of SVM using a sigmoid kernel is summarized in the table~\ref{tab:sigwp} with considering permanent faults in the sample population, and the same without considering permanents is summarized in the table~\ref{tab:sigwop}.

Again, permanent fault classification rates for permanent faults is fairly high for sigmoid kernel also. The Intermittent fault classification accuracy is comparatively low (except p100k) despite high cross validation accuracy. This coupled with high transient fault classification accuracy indicate that, sigmoid kernel is biased towards transient faults. This has also resulted in a degradation of accuracy levels when intermittents were injected with transient noise.

\begin{table}[h]

	\captionsetup{justification=centering}
\begin{tabular}{ccccccc}
\hline
\multirow{3}{*}{Circuit} & \multicolumn{6}{c}{Accuracy (\%)}\\ \cline{2-7} 
                         & \multirow{2}{*}{Cross-validation} & \multicolumn{2}{c}{Permanent} & \multicolumn{2}{c}{Intermittent} & \multirow{2}{*}{Transient} \\ \cline{3-6}
                         &                                   & w/o noise     & with noise    & w/o noise      & with noise      &                            \\ \hline
p45k                     & 89.55                             & 97.50         & 97.50         & 77.08          & 60.41           & 95.91                      \\
p100k                    & 85.08                             & 100           & 100           & 95.23          & 52.38           & 71.42                      \\
p141k                    & 84.17                             & 100           & 100           & 57.14          & 57.14           & 100                        \\
p267k                    & 82.46                             & 100           & 98.27         & 42.50          & 42.50           & 100                        \\
p279k                    & 89.29                             & 98.30         & 96.55         & 81.57          & 52.63           & 87.76                      \\
p295k                    & 85.09                             & 98.21         & 98.21         & 69.76          & 65.11           & 100                       \\
\hline                                                     
\end{tabular}
\caption {Classification accuracy for sigmoid kernel}
\label{tab:sigwp}
\end{table}

The kernel became more biased towards transients when permanent fault examples were removed from the sample population. Transient fault classification accuracy was observed to increase to near 100\%. Also, the intermittent classification accuracy also increased at the same time, with and without transient noise.

\begin{table}[h]
\captionsetup{justification=centering}
\begin{tabular}{ccccc}
\hline
\multirow{3}{*}{Circuit} & \multicolumn{4}{c}{Accuracy (\%)}                                                                 \\ \cline{2-5} 
                         & \multirow{2}{*}{Cross-validation} & \multicolumn{2}{c}{Intermittent} & \multirow{2}{*}{Transient} \\ \cline{3-4}
                         &                                   & w/o noise      & with noise      &                            \\ \hline
p45k                     & 84.46                             & 68.75          & 60.41           & 100                        \\
p100k                    & 80.00                             & 81.39          & 52.38           & 100                        \\
p141k                    & 80.04                             & 78.57          & 73.80           & 100                        \\
p267k                    & 75.17                             & 75.00          & 55.00           & 100                        \\
p279k                    & 83.41                             & 81.57          & 52.63           & 91.83                      \\
p295k                    & 92.28                             & 93.02          & 86.04           & 91.83                     \\
\hline
\end{tabular}
\caption {Classification accuracy for sigmoid kernel, without permanent faults}
\label{tab:sigwop}
\end{table}

This set of experiments show that the relative accuracy levels for the fault classification actually increase if the sample population and test data is pre-screened for permanents faults. Hence for the rest of the evaluation,  all experiments are done by considering that the sample population consists of intermittent and transient faults only.

\subsection{Analysis of intermittent fault classification}
\label{sec:wp:intanalysis}
The first set of experiments reveal that the classification accuracy for intermittent faults comparatively low as compared to other fault types. Hence more detailed analysis is done, which can provide some hints about the improvement of intermittent fault classification accuracy. First, an analysis is done about how fault activation rates might be affecting classification accuracy. Table~\ref{tab:intdetail} summarizes the result of this analysis and it notes accuracy results for different intermittent fault activation rates, for examples in the the test dataset. For this analysis, a subset of circuits were analyzed using all four kernels of \texttt{LIBSVM}.

\begin{table}[h]
\captionsetup{justification=centering}
\resizebox{\textwidth}{!}{%
\begin{tabular}{cccccccc}
\hline
\multirow{3}{*}{Circuit} & \multirow{3}{*}{Kernel} & \multicolumn{6}{c}{Classification accuracy (\%)}\\ \cline{3-8} 
                         &                         & \multicolumn{3}{c}{Intermittent faults without noise} & \multicolumn{3}{c}{Intermittent faults with noise} \\ \cline{3-8} 
                         &                         & rate = 0.1    & rate = 0.01    & rate = 0.001   & rate = 0.1   & rate = 0.01   & rate = 0.001  \\ \hline
\multirow{4}{*}{p141k}   & linear                  & 100           & 81.25          & 25.00          & 100          & 75.00         & 25.00         \\
                         & polynomial              & 100           & 81.25          & 37.5           & 100          & 75.00         & 25.00         \\
                         & RBF                     & 100           & 81.25          & 50.00          & 100          & 81.25         & 37.5          \\
                         & sigmoid                 & 100           & 81.25          & 25.00          & 100          & 68.75         & 25.00         \\
\hline
\multirow{4}{*}{p267k}   & linear                  & 94.73         & 82.35          & 20.00          & 89.47        & 35.29         & 0.00          \\
                         & polynomial              & 94.73         & 88.23          & 20.00          & 94.73        & 64.70         & 20.00         \\
                         & RBF                     & 94.73         & 88.2           & 20.00          & 94.73        & 58.82         & 20.00         \\
                         & sigmoid                 & 94.73         & 70.58          & 20.00          & 89.47        & 35.29         & 0.00          \\

\hline
\multirow{4}{*}{p295k}   & linear                  & 100           & 88.23          & 77.77          & 100          & 88.23         & 33.33         \\
                         & polynomial              & 100           & 94.11          & 77.77          & 100          & 100           & 33.33         \\
                         & RBF                     & 100           & 94.117         & 77.77          & 100          & 100           & 33.33         \\
                         & sigmoid                 & 100           & 88.23          & 77.77          & 100          & 76.47         & 22.22     	\\
\hline   
\end{tabular}
}
\caption{Accuracy of intermittent fault classification for different fault rates}
\label{tab:intdetail}
\end{table}

The analysis shows that, accuracy levels are fairly high for high fault activation rates of intermittent faults, but go on deteriorating for lower values of fault activation rates. The main reason this behavior is that, at lower fault activation rates of $10^{-3}$, the behavior of transient faults (activation rates for which are $10^{-2}$ to $10^{-4}$) matches the behavior of intermittents, and the learning algorithm fails to separate both from each other. This suggests that, instances where intermittent failures occurred only once and did not repeat themselves, are not able to convey enough information in their extracted features. Hence no classification algorithm will be able to improve on these figures significantly. To further elaborate this, accuracy analysis for the same circuits from table~\ref{tab:intdetail} is done by removing those intermittent faults from the test set, whose reproducibility ($\epsilon$) is less than one. The results of this analysis are presented in table~\ref{tab:intsweep}.

\begin{table}[h]
\begin{tabular}{ccccc}
\hline
\multirow{3}{*}{Circuit} & \multirow{3}{*}{Kernel} & \multicolumn{3}{c}{Classification accuracy (\%)}              \\ \cline{3-5} 
                         &                         & \multicolumn{2}{c}{Intermittent} & \multirow{2}{*}{Transient} \\ \cline{3-4}
                         &                         & w/o noise      & with noise      &                            \\ \hline
\multirow{4}{*}{p141k}   & linear                  & 100            & 100             & 100                        \\
                         & polynomial              & 100            & 100             & 100                        \\
                         & RBF                     & 100            & 100             & 100                        \\
                         & sigmoid                 & 100            & 100             & 100                        \\
\hline
\multirow{4}{*}{p267k}   & linear                  & 100            & 100             & 100                        \\
                         & polynomial              & 100            & 100             & 93.87                      \\
                         & RBF                     & 100            & 100             & 95.91                      \\
                         & sigmoid                 & 100            & 100             & 100                        \\
\hline
\multirow{4}{*}{p295k}   & linear                  & 100            & 100             & 100                        \\
                         & polynomial              & 100            & 100             & 93.87                      \\
                         & RBF                     & 100            & 100             & 91.83                      \\
                         & sigmoid                 & 100            & 100             & 91.83                     \\
\hline
\end{tabular}
\caption{Improvement in accuracy levels after removing intermittents with $\epsilon = 1$}
\label{tab:intsweep}
\end{table}

The analysis from table~\ref{tab:intsweep} shows that, if we remove such data from the test set, classification of intermittent faults is accurate upto 100\%. The accuracy of transient fault classification remains the same, as the same classifier model and the same test dataset for transient faults is used as earlier experiments.

\section{With class-weight optimization}
\label{sec:ww}
The other possibility than the one suggested in the section~\ref{sec:wp:intanalysis} is to balance the training dataset. An unbalanced dataset is a possibility in our case, as we have set a tolerance of $\pm$ 75 on training examples of each class, as explained in the section~\ref{sec:gsp:configuration}. Class-weights are used while training the SVM by adding a penalty to samples of particular class, to compensate for the unavailability of enough positive examples for that class.

\texttt{LIBSVM} provides a functionality to adjust individual class weights, to bias classification towards a specified class using \texttt{-wi} switch. We can use this functionality to:
\begin{itemize}
  \item Fine-tune $C$ values for optimal classification. This way we can have a classifier which can separate most of examples in dataset.
  \item Optimize classifier for classifying one type of faults more accurately than the other.This is particularly useful the user is more interested in optimizing for the quality, then he may wish to reject all chips with a slightest possibility of intermittent failure. In this case classifier can be optimized for classifying intermittent faults more accurately. On the other hand, if user wants to optimize for quality, classifier can be tuned to classify transient faults more accurately.
\end{itemize}  

Table~\ref{tab:optp267k} shows accuracy values, when optimizations mentioned above are performed.

\begin{table}[h]
\captionsetup{justification=centering}
\begin{tabular}{cccccc}
\hline
\multirow{3}{*}{Kernel}     & \multirow{3}{*}{Optimization Criterion} & \multirow{3}{*}{Class-weights \{\texttt{I,T}\}} & \multicolumn{3}{c}{Classification accuracy (\%)}              \\ \cline{4-6} 
                            &                                         &                                                   & \multicolumn{2}{c}{Intermittent} & \multirow{2}{*}{Transient} \\ \cline{4-5}
                            &                                         &                                                   & w/o noise      & with noise      &                            \\ \hline
\multirow{3}{*}{linear}     & Intermittent                            & \{1.5,1\}                                         & 100            & 100             & 22.44                      \\
                            & Transient                               & \{1,1\}                                           & 80.00          & 55.00           & 100                        \\

                            & All                               & \{1.9,1.8\}                                           & 72.5          & 67.5           & 95.91                        \\
\hline
\multirow{3}{*}{polynomial} & Intermittent                            & \{1.85,1.05\}                                     & 97.50          & 80.00           & 20.00                      \\
                            & Transient                               & \{1.3,1.95\}                                      & 82.50          & 67.50           & 97.95                      \\

                            & All                               & \{1.9,1.25\}                                           & 95.00          & 78.00           & 77.5                        \\
\hline
\multirow{3}{*}{RBF}        & Intermittent                            & \{1.75,1.15\}                                     & 95.00          & 72.50           & 89.79                      \\
                            & Transient                               & \{1.2,1.3\}                                       & 72.50          & 70.00           & 95.91                      \\

                            & All                               & \{1,1\}                                           & 82.50          & 70.00           & 92.34                        \\
\hline
\multirow{3}{*}{sigmoid}    & Intermittent                            & \{1.7,1.1\}                                       & 85.00          & 77.50           & 65.30                      \\
                            & Transient                               & \{1.9,1.8\}                                       & 82.5           & 57.5            & 100                       \\

                            & All                               & \{1.7,1.75\}                                           & 82.50          & 47.00           & 100                        \\
\hline
\end{tabular}
\caption{Accuracy after class-weight optimization for p267k}
\label{tab:optp267k}
\end{table}

To find out optimal values of weights in the case of optimization for both intermittent and transient faults at the same time, an exhaustive search of combinations of class-weights for \texttt{I} and \texttt{T} classes is performed. For experimental purpose, a limited search of region [1,2] with steps of 0.5 for combinations of class-weights is performed. Heat-maps for this search are shown in figure~\ref{fig:heatmap} for p267k. Using these heat-maps the weight combination were set up for the analysis presented above.

Optimization for \enquote{All} labels is done by considering weights corresponding to the highest cross-validation accuracy. After class-weight optimization, accuracy levels are observed to be improved as compared to the same analysis done in section~\ref{sec:wp}.


\begin{figure}
        \centering
			\captionsetup{justification=centering}
        \begin{subfigure}[h]{0.45\linewidth}
                \includegraphics[scale=0.25]{figures/lin256i.png}
                \caption{Intermittent fault accuracy, linear kernel}
        \end{subfigure}
        \begin{subfigure}[h]{0.45\linewidth}
                \includegraphics[scale=0.25]{figures/lin256t.png}
                \caption{Transient fault accuracy, linear kernel}
        \end{subfigure}

			\begin{subfigure}[h]{0.45\linewidth}
                \includegraphics[scale=0.25]{figures/poly256i.png}
                \caption{Intermittent fault accuracy, polynomial kernel}
        \end{subfigure}
        \begin{subfigure}[h]{0.45\linewidth}
                \includegraphics[scale=0.25]{figures/poly256i.png}
                \caption{Transient fault accuracy, polynomial kernel}
        \end{subfigure}

        \begin{subfigure}[h]{0.45\linewidth}
                \includegraphics[scale=0.25]{figures/rbf256i.png}
                \caption{Intermittent fault accuracy, RBF kernel}
        \end{subfigure}
			\begin{subfigure}[h]{0.45\linewidth}
                \includegraphics[scale=0.25]{figures/rbf256t.png}
                \caption{Transient fault accuracy, RBF kernel}
        \end{subfigure}

        \begin{subfigure}[h]{0.45\linewidth}
                \includegraphics[scale=0.25]{figures/sig256i.png}
                \caption{Intermittent fault accuracy, sigmoid kernel}
        \end{subfigure}
			\begin{subfigure}[h]{0.45\linewidth}
                \includegraphics[scale=0.25]{figures/sig256t.png}
                \caption{Transient fault accuracy, sigmoid kernel}
        \end{subfigure}

        \caption{Plots of accuracy by varying class-weights for p256k}
        \label{fig:heatmap} 
\end{figure}



\section{Classification using extrapolation of known training datasets}
So far the evaluation of test data is done using a sample population of the same circuit type. In this section two other possibilities are considered. First, to use a sample population of a known circuit type, to classify the data of another circuit type. Second, to have a single training dataset, and use a classifier model built from this to predict example data of a circuit type, whose sample population is present in this \emph{universal} dataset.

\subsection{Using single known model for classification}
In practical situations, we might not have the necessary training data for a new product beforehand, especially at the start of production. This section explores possibility of extrapolating known datasets for new circuit types. This set of experiments are done using the training data of p295k to predict test sets of other circuit types. p295k is chosen as sample population as it provided with the relatively most accurate results in the first set of experiments. The results obtained are summarized in the table~\ref{tab:univ295k}.

\begin{table}[h]
\captionsetup{justification=centering}
\resizebox{\textwidth}{!}{%
\begin{tabular}{cccccc}
\hline
\multirow{3}{*}{Kernel}     & \multirow{3}{*}{Cross-validation accuracy (\%)} & \multirow{3}{*}{Circuit} & \multicolumn{3}{c}{Accuracy (\%)}                             \\ \cline{4-6} 
                            &                                                 &                          & \multicolumn{2}{c}{Intermittent} & \multirow{2}{*}{Transient} \\ \cline{4-5}
                            &                                                 &                          & without noise    & with noise    &                            \\ \hline
\multirow{5}{*}{Linear}     & \multirow{5}{*}{89.17}                          & p45k                     & 77.08            & 72.91         & 95.91                      \\
                            &                                                 & p100k                    & 71.42            & 64.28         & 75.51                      \\
                            &                                                 & p141k                    & 78.57            & 73.08         & 100                        \\
                            &                                                 & p267k                    & 77.5             & 62.5          & 100                        \\
                            &                                                 & p279k                    & 78.94            & 57.89         & 97.95                      \\
\hline
\multirow{5}{*}{Polynomial} & \multirow{5}{*}{90.78}                          & p45k                     & 79.16            & 81.63         & 75.51                      \\
                            &                                                 & p100k                    & 78.57            & 76.19         & 73.46                      \\
                            &                                                 & p141k                    & 88.09            & 80.95         & 91.83                      \\
                            &                                                 & p267k                    & 82.5             & 77.5          & 87.75                      \\
                            &                                                 & p279k                    & 78.94            & 76.31         & 97.95                      \\
\hline
\multirow{5}{*}{RBF}        & \multirow{5}{*}{92.28}                          & p45k                     & 79.16            & 81.25         & 73.46                      \\
                            &                                                 & p100k                    & 78.57            & 73.8          & 71.42                      \\
                            &                                                 & p141k                    & 88.09            & 80.95         & 97.95                      \\
                            &                                                 & p267k                    & 82.5             & 72.5          & 89.79                      \\
                            &                                                 & p279k                    & 78.94            & 71.05         & 100                        \\
\hline
\multirow{5}{*}{Sigmoid}    & \multirow{5}{*}{85.64}                          & p45k                     & 79.16            & 81.25         & 73.46                      \\
                            &                                                 & p100k                    & 78.57            & 73.8          & 71.42                      \\
                            &                                                 & p141k                    & 88.09            & 80.95         & 97.95                      \\
                            &                                                 & p267k                    & 82.5             & 72.5          & 89.79                      \\
                            &                                                 & p279k                    & 78.94            & 71.79         & 100                       \\
\hline
\end{tabular}
}
\caption{Accuracy by extrapolating training dataset of p295k to test other circuits}
\label{tab:univ295k}
\end{table}

Comparison with results from those obtained in the section~\ref{sec:wp}, shows an increase in accuracy levels. This can be attributed to two factors, a comparatively balanced dataset of p295k and high cross-validation accuracy levels of this dataset. Hence it is observed that, using a good-quality known dataset, examples from other circuit types can be classified with acceptable accuracy levels.

\subsection{Using a universal training set for classification}

In this set of experiments, a single universal sample population is built using sample populations of individual circuit types. The aim of this experiment to check a possibility of building an incremental universal dataset, which is able to classify data from any of known circuit types. Table~\ref{tab:universal} summarizes results observed.

\begin{table}[h]
\captionsetup{justification=centering}
\resizebox{\textwidth}{!}{%
\begin{tabular}{cccccc}
\hline
\multirow{3}{*}{Kernel}     & \multirow{3}{*}{Cross-validation accuracy (\%)} & \multirow{3}{*}{Circuit} & \multicolumn{3}{c}{Classification accuracy (\%)}              \\ \cline{4-6} 
                            &                                                 &                          & \multicolumn{2}{c}{Intermittent} & \multirow{2}{*}{Transient} \\ \cline{4-5}
                            &                                                 &                          & without noise    & with noise    &                            \\ \hline
\multirow{6}{*}{Linear}     & \multirow{6}{*}{82.84}                          & p45k                     & 66.66            & 64.58         & 100                        \\
                            &                                                 & p100k                    & 66.66            & 61.9          & 89.79                      \\
                            &                                                 & p141k                    & 80.95            & 76.19         & 100                        \\
                            &                                                 & p267k                    & 70               & 65            & 95.91                      \\
                            &                                                 & p279k                    & 71.05            & 60.52         & 97.95                      \\
                            &                                                 & p295k                    & 81.39            & 79.06         & 100                        \\
\hline
\multirow{6}{*}{Polynomial} & \multirow{6}{*}{86.52}                          & p45k                     & 75               & 25            & 95.91                      \\
                            &                                                 & p100k                    & 98.57            & 66.66         & 89.79                      \\
                            &                                                 & p141k                    & 88.09            & 78.57         & 100                        \\
                            &                                                 & p267k                    & 82.5             & 67.5          & 97.95                      \\
                            &                                                 & p279k                    & 76.31            & 63.15         & 100                        \\
                            &                                                 & p295k                    & 93.02            & 81.39         & 95.91                      \\
\hline
\multirow{6}{*}{RBF}        & \multirow{6}{*}{86.58}                          & p45k                     & 79.16            & 77.08         & 89.79                      \\
                            &                                                 & p100k                    & 78.57            & 69.04         & 87.75                      \\
                            &                                                 & p141k                    & 88.09            & 78.57         & 100                        \\
                            &                                                 & p267k                    & 82.5             & 70            & 93.87                      \\
                            &                                                 & p279k                    & 78.94            & 68.42         & 100                        \\
                            &                                                 & p295k                    & 93.02            & 86.04         & 97.95                      \\
\hline
\multirow{6}{*}{Sigmoid}    & \multirow{6}{*}{83.71}                          & p45k                     & 81.25            & 64.58         & 79.59                      \\
                            &                                                 & p100k                    & 83.33            & 66.66         & 83.67                      \\
                            &                                                 & p141k                    & 88.09            & 71.42         & 81.63                      \\
                            &                                                 & p267k                    & 87.5             & 72.5          & 69.38                      \\
                            &                                                 & p279k                    & 81.57            & 60.52         & 79.59                      \\
                            &                                                 & p295k                    & 97.67            & 86.04         & 85.71     \\
\hline                
\end{tabular}
}
\caption{Accuracy using single universal training dataset}
\label{tab:universal}
\end{table}

The universal dataset is also able to classify faults fairly accurately. Using polynomial and RBF kernels, accuracy levels of both intermittent and transient fault classification increased considerably at the same time. This is mainly because of an increase in the sample population providing  higher number of positive and negative examples of both classes and eventually resulting in better fitting of hypothesis class. A further increase in accuracy levels, for both or any one of fault types is possible using techniques discussed in section~\ref{sec:ww}.



%Die Angabe des schlauen Spruchs auf diesem Wege funtioniert nur,
%wenn keine Änderung des Kapitels mittels den in preambel/chapterheads.tex
%vorgeschlagenen Möglichkeiten durchgeführt wurde.
\chapter{Conclusion}
\label{chap:chapter7}
%\vspace{-3cm}
%\vspace{2cm}


%\input{content/kapitel2}
%\input{content/zusammenfassung_und_ausblick}
%
%
%\renewcommand{\appendixtocname}{Anhang}
%\renewcommand{\appendixname}{Anhang}
%\renewcommand{\appendixpagename}{Anhang}
\appendix
%\input{content/anhang}
%\printindex
%\bibliographystyle{alphadin}
\ifdeutsch
\bibliographystyle{bibliography/IAASde} %f"ur deutsche Texte
\else
\bibliographystyle{bibliography/IAAS} %f"ur englische Texte
\fi
\bibliography{bibliography/bibliography}
\ifdeutsch
Alle URLs wurden zuletzt am 23.\,04.\,2014 geprüft.
\else
All links were last followed on April 23, 2014.
\fi

%\backmatter 
\pagestyle{empty}
\renewcommand*{\chapterpagestyle}{empty}
\Versicherung
\end{document}
