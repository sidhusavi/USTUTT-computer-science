\chapter{Machine Learning}
\label{chap:chap3}

An \emph{algorithm} is set of instructions used to convert input values to output, based on certain rules. Consider an example where we need to find all even numbers from dataset. Here, we can set up a \emph{rule} that if number is completely divisible by two then it should be included in the output dataset, otherwise not. Naturally, as there can be more than one way to solve a problem, there can be more than one algorithm to solve it. However there are certain examples where formation of set of rule is practically infeasible. For example, consider handwriting recognition software used to scan handwritten forms. Figure illustrates problem at hand, where a simple character can be written in a number of ways. It is interesting to note that humans are able to read this data without trouble, but it is really difficult express a certain rules which will result in accurate recognition with help of an algorithm. Machine learning is employed in such cases. Specifically \emph{Machine Learning} (ML) is programming computers to optimize a performance criterion (e.g. character recognition) using example data or past experience \cite{Alpaydin2004}. 

%figure

The "example data" with its \emph{label} is collectively called as \emph{training data}, and it is used to teach machine learning how the character with given label looks like, so that ML can recognize when it encounters similar data in future. Machine learning can be applied in wide range of applications where it is not possible to express expertise but a large amount of sample data is available. Typical applications of machine learning include computer vision, pattern recognition, spam filtering, search result optimization etc. 
