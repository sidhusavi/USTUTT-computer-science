%Die Angabe des schlauen Spruchs auf diesem Wege funtioniert nur,
%wenn keine Änderung des Kapitels mittels den in preambel/chapterheads.tex
%vorgeschlagenen Möglichkeiten durchgeführt wurde.
\chapter{Evaluation of results}
\label{chap:chapter6}
%\vspace{-3cm}
%\vspace{2cm}
This chapter presents results of classification using experimental setup described in chapter~\ref{chap:chapter6}. The term \emph{Classification accuracy} of a class used in chapter is defined as,

\[ \begin{array}{ll} accuracy(r_i) = \frac{\mbox{Samples correctly predicted as $r_i$}}
								{\mbox{Total samples with label $r_i$}}
							\times 100 & 
							\forall \,r_i \in \{\mbox{\texttt{P,I,T}}\} 
	\end{array}\]

Where, $r$ is the label of test sample $x=\{\boldsymbol{x},r\}$.

In the presented evaluation, cross-validation accuracy is also noted for each set of experiments, as this is value of accuracy during self-validation of much larger training set.

In the first part analyzes classification efficiency after grid search for optimal values of $C$ and $\gamma$, but does not consider any other optimization, detailed account of results are presented for each kernel type available in \texttt{LIBSVM}. Second part improvises the first, and results are noted after optimization of so-called \emph{class-weights}. Third part explores two possibilities -that of using one  of using single prediction model for predicting all types of circuits, and second to extrapolating known model to predict new circuit types. The last part tries to evaluate impact on yield and quality of final product after application of suggested method.

\section{Classification of permanent faults}

Consider a special case, where $\epsilon = Test Runs$. In our classification problem this can mean,
\begin{enumerate}
  \item It is a permanent fault.
  \item A rare case of highly repetitive intermittent fault, and can be said to be \enquote{critical}.
  \item An extremely rare transient fault, as it happened at the same location and for same input test pattern, in all of the test runs.
\end{enumerate}

In the other direction, for permanent faults $\epsilon = Test Runs$ holds true, except a rare possibility that transient noise affects all the test patterns, which have resulted in failure at POs, in at least one of test runs. This is also an extremely rare possibility. Hence we assume that, a fault is permanent if and only if $\epsilon = Test Runs$.

If we go back to section~\ref{sec:secfs} and observe the plot for $\epsilon$ in figure~\ref{fig:epsilonp45k}, we can observe our assumption to be fairly accurate. Hence if we do a slight modification in our experimental setup and already remove faults where $\epsilon = Test Runs$, and classify them as permanent faults, we can achieve 100\% permanent fault classification, and we would be left with a binary classification problem between transient and intermittent faults. 

\section{Classification without class-weight optimization}
First set of experiments consist of evaluation of accuracy levels without assigning class weights \emph{i.e.} class-weights are assumed to be \{1,1,1\} for permanent, intermittent and transient faults respectively. The experiments are carried out for variety of circuits ranging from simple p45k to highly complex p295k. The experiments are repeated for each type of kernel provided by \texttt{LIBSVM}. 

This set of experiments consist of two rounds each for each kernel, first one includes permanent faults in sample population, the other one without considering them. A round with permanents is carried out only to evaluate change in accuracy levels of classification.

\subsection{Linear Kernel}
Accuracy of SVM using linear kernel is summarized in table~\ref{tab:linwp} with considering permanent faults in sample population, and the same without considering permanents is summarized in table~\ref{tab:linwop}.
\begin{table}[h]

	\captionsetup{justification=centering}
\begin{tabular}{ccccccc}
\hline
\multicolumn{1}{c}{\multirow{3}{*}{Circuit}} & \multicolumn{6}{c}{Accuracy (\%)}\\ \cline{2-7} 
\multicolumn{1}{c}{}                         & \multicolumn{1}{c}{\multirow{2}{*}{Cross-validation}} & \multicolumn{2}{c}{Permanent} & \multicolumn{2}{c}{Intermittent} & \multicolumn{1}{c}{\multirow{2}{*}{Transient}} \\ \cline{3-6}
                                             &                                                       & w/o noise     & with noise    & w/o noise      & with noise      &                                                \\ \hline
p45k                                         & 90.14                                                 & 87.50         & 87.50         & 77.50          & 64.58           & 93.90                                          \\
p100k                                        & 87.38                                                 & 100           & 100           & 80.95          & 57.14           & 95.91                                          \\
p141k                                        & 83.89                                                 & 98.21         & 98.21         & 61.90          & 57.14           & 100                                            \\
p267k                                        & 85.12                                                 & 100           & 100           & 55.00          & 52.50           & 100                                            \\
p279k                                        & 89.52                                                 & 98.30         & 96.61         & 78.94          & 52.50           & 89.79                                          \\
p295k                                        & 89.84                                                 & 100           & 100           & 81.39          & 65.11           & 100   \\
\hline                                        
\end{tabular}
\caption {Classification accuracy for linear kernel}
\label{tab:linwp}
\end{table}

In both cases, with and without permanent faults, it can be observed that, permanent faults are being relatively accurately being classified and background transient noise seems to have almost no effect on classification accuracy of permanent faults. Intermittent faults in both cases have relatively low accuracy levels and tend to deteriorate severely in presence of transient noise. 


\begin{table}[h]
\captionsetup{justification=centering}
\begin{tabular}{ccccc}
\hline
\multirow{3}{*}{Circuit} & \multicolumn{4}{c}{Accuracy (\%)}\\ \cline{2-5} 
                         & \multirow{2}{*}{Cross-validation} & \multicolumn{2}{c}{Intermittent} & \multirow{2}{*}{Transient} \\ \cline{3-4}
                         &                                   & w/o noise      & with noise      &                            \\ \hline
p45k                     & 87.81                             & 72.91          & 70.08           & 97.95                      \\
p100k                    & 81.43                             & 69.05          & 59.52           & 97.96                      \\
p141k                    & 84.13                             & 78.57          & 76.19           & 100                        \\
p267k                    & 77.09                             & 80.00          & 55.00           & 100                        \\
p279k                    & 84.65                             & 78.94          & 52.63           & 91.83                      \\
p295k                    & 89.17                             & 90.69          & 81.39           & 100           \\
\hline            
\end{tabular}
\caption {Classification accuracy for linear kernel, without permanent faults}
\label{tab:linwop}
\end{table}

After removing permanents from the sample population, classification accuracies are observed to improve, with an exception of p100k. However, accuracy levels are observed to increase for intermittents with noise even in the case of p100k. Classification accuracy of transient faults has also increased. Cross-validation accuracy levels have decreased, as large chuck of correctly classified data in earlier case was permanent faults. Comparatively lower accuracy levels for intermittent faults and almost 100\% classification rate for transients suggests that, linear kernel is more biased towards transient faults.

\subsection{Polynomial Kernel}
\begin{table}[h]

	\captionsetup{justification=centering}
\begin{tabular}{ccccccc}
\hline
\multirow{3}{*}{Circuit} & \multicolumn{6}{c}{Accuracy (\%)}\\ \cline{2-7} 
                         & \multirow{2}{*}{Cross-validation} & \multicolumn{2}{c}{Permanent} & \multicolumn{2}{c}{Intermittent} & \multirow{2}{*}{Transient} \\ \cline{3-6}
                         &                                   & w/o noise     & with noise    & w/o noise      & with noise      &                            \\ \hline
p45k                     & 90.92                             & 87.50         & 89.58         & 68.75          & 70.83           & 83.67                      \\
p100k                    & 89.70                             & 100           & 97.82         & 69.04          & 66.66           & 93.87                      \\
p141k                    & 87.94                             & 100           & 100           & 73.80          & 71.42           & 95.00                      \\
p267k                    & 87.34                             & 100           & 96.55         & 82.50          & 60.00           & 85.71                      \\
p279k                    & 91.42                             & 98.30         & 96.55         & 81.57          & 73.68           & 91.83                      \\
p295k                    & 90.78                             & 100           & 98.21         & 76.09          & 72.09           & 95.91      \\
\hline                                                     
\end{tabular}
\caption {Classification accuracy for polynomial kernel}
\label{tab:polywp}
\end{table}

Accuracy of SVM using polynomial kernel is summarized in table~\ref{tab:polywp} with considering permanent faults in sample population, and the same without considering permanents is summarized in table~\ref{tab:polywop}.

Accuracy of classification of permanent faults, without noise in case of polynomial kernel is observed to be even higher than linear kernel. For permanents with noise, classification rates are almost similar. Intermittent faults with permanents in first round, without noise are deteriorating in some cases, while improving significantly in others. With noise, however, accuracy of intermittent fault classification is improved. Classification accuracy of transient faults is decreased as compared to linear kernel.

\begin{table}[h]

	\captionsetup{justification=centering}
\begin{tabular}{ccccc}
\hline
\multirow{3}{*}{Circuit} & \multicolumn{4}{c}{Accuracy (\%)}\\ \cline{2-5} 
                         & \multirow{2}{*}{Cross-validation} & \multicolumn{2}{c}{Intermittent} & \multirow{2}{*}{Transient} \\ \cline{3-4}
                         &                                   & w/o noise      & with noise      &                            \\ \hline
p45k                     & 87.81                             & 81.25          & 72.91           & 87.76                      \\
p100k                    & 85.00                             & 78.57          & 66.66           & 89.79                      \\
p141k                    & 85.57                             & 80.95          & 76.19           & 100                        \\
p267k                    & 81.38                             & 82.50          & 72.50           & 93.87                      \\
p279k                    & 85.57                             & 78.94          & 73.68           & 93.87                      \\
p295k                    & 92.28                             & 93.02          & 86.04           & 93.87                      \\
\hline
\end{tabular}
\caption {Classification accuracy for polynomial kernel, without permanent faults}
\label{tab:polywop}
\end{table}

In second round, after removing permanents, a huge improvement in accuracy of intermittent faults is observed as compared to the first round. The results are also better than those for linear kernel.  A marginal improvement in transient fault classification accuracy is also observed as compared to the first round. However, overall figures for transients are lower than those for linear kernel. Observations suggest that polynomial kernel is slightly biased towards intermittents.


\subsection{RBF Kernel}

Accuracy of SVM using Radial Basis Function (RBF) kernel is summarized in table~\ref{tab:rbfwp} with considering permanent faults in sample population, and the same without considering permanents is summarized in table~\ref{tab:rbfwop}.

\begin{table}[h]

	\captionsetup{justification=centering}
\begin{tabular}{ccccccc}
\hline
\multirow{3}{*}{Circuit} & \multicolumn{6}{c}{Accuracy (\%)}\\ \cline{2-7} 
                         & \multirow{2}{*}{Cross-validation} & \multicolumn{2}{c}{Permanent} & \multicolumn{2}{c}{Intermittent} & \multirow{2}{*}{Transient} \\ \cline{3-6}
                         &                                   & w/o noise     & with noise    & w/o noise      & with noise      &                            \\ \hline
p45k                     & 91.21                             & 100           & 100           & 66.66          & 64.58           & 91.83                      \\
p100k                    & 87.36                             & 100           & 100           & 64.28          & 61.94           & 93.87                      \\
p141k                    & 88.63                             & 100           & 100           & 73.80          & 71.42           & 93.87                      \\
p267k                    & 87.27                             & 100           & 98.27         & 80.00          & 72.50           & 93.87                      \\
p279k                    & 89.52                             & 98.30         & 96.55         & 78.92          & 71.05           & 93.87                      \\
p295k                    & 91.00                             & 100           & 100           & 79.06          & 74.41           & 95.91                     \\
\hline                                                     
\end{tabular}
\caption {Classification accuracy for RBF kernel}
\label{tab:rbfwp}
\end{table}
Classification rates for permanent faults, with and without noise, are higher for RBF kernel as compared to others. Like in other cases, injection of transient noise has affected accuracy of intermittent fault classification, significantly. Accuracy figures for both, intermittent faults with and without noise, are lower as compared to previous two kernels. Transient fault classification accuracy shows marginal improvement over polynomial kernel, but is still lower than the linear kernel.

Intermittent fault classification accuracy is increased for simpler circuits, but shows no change for more complex circuits. However, for intermittent faults injected with noise, accuracy is improved as compared to the first round, but is till lower than other kernel types. A slight increase in transient fault classification accuracy is also observed. 

A notable observation in case of RBF kenrel is its comparatively higher cross-validation accuracy figures and lower test set classification accuracy, an indication that the kernel may be overfitting.

\begin{table}[h]
	\captionsetup{justification=centering}
\begin{tabular}{ccccc}
\hline
\multirow{3}{*}{Circuit} & \multicolumn{4}{c}{Accuracy (\%)}                                                                 \\ \cline{2-5} 
                         & \multirow{2}{*}{Cross-validation} & \multicolumn{2}{c}{Intermittent} & \multirow{2}{*}{Transient} \\ \cline{3-4}
                         &                                   & w/o noise      & with noise      &                            \\ \hline
p45k                     & 87.56                             & 77.08          & 70.83           & 91.83                      \\
p100k                    & 83.33                             & 78.57          & 64.28           & 89.79                      \\
p141k                    & 86.53                             & 83.33          & 80.95           & 100                        \\
p267k                    & 80.19                             & 82.50          & 70.00           & 95.91                      \\
p279k                    & 85.64                             & 78.92          & 76.31           & 95.91                      \\
p295k                    & 85.64                             & 79.06          & 76.31           & 95.91  						  \\
\hline
\end{tabular}
\caption {Classification accuracy for RBF kernel, without permanent faults}
\label{tab:rbfwop}
\end{table}


\subsection{Sigmoid Kernel}

Accuracy of SVM using Radial Basis Function (RBF) kernel is summarized in table~\ref{tab:sigwp} with considering permanent faults in sample population, and the same without considering permanents is summarized in table~\ref{tab:sigwop}.

Again, the permanent fault classification rates for permanent faults is fairly high for sigmoid kernel also. The Intermittent fault classification accuracy is comparatively low (except p100k) despite high cross validation accuracy. This coupled with high transient fault classification accuracy indicate that sigmoid kernel is biased towards transient faults. This has also resulted in degradation of accuracy levels when intermittents were injected with transient noise.

\begin{table}[h]

	\captionsetup{justification=centering}
\begin{tabular}{ccccccc}
\hline
\multirow{3}{*}{Circuit} & \multicolumn{6}{c}{Accuracy (\%)}\\ \cline{2-7} 
                         & \multirow{2}{*}{Cross-validation} & \multicolumn{2}{c}{Permanent} & \multicolumn{2}{c}{Intermittent} & \multirow{2}{*}{Transient} \\ \cline{3-6}
                         &                                   & w/o noise     & with noise    & w/o noise      & with noise      &                            \\ \hline
p45k                     & 89.55                             & 97.50         & 97.50         & 77.08          & 60.41           & 95.91                      \\
p100k                    & 85.08                             & 100           & 100           & 95.23          & 52.38           & 71.42                      \\
p141k                    & 84.17                             & 100           & 100           & 57.14          & 57.14           & 100                        \\
p267k                    & 82.46                             & 100           & 98.27         & 42.50          & 42.50           & 100                        \\
p279k                    & 89.29                             & 98.30         & 96.55         & 81.57          & 52.63           & 87.76                      \\
p295k                    & 85.09                             & 98.21         & 98.21         & 69.76          & 65.11           & 100                       \\
\hline                                                     
\end{tabular}
\caption {Classification accuracy for sigmoid kernel}
\label{tab:sigwp}
\end{table}

The kernel became more biased towards transients when permanent fault examples were removed from sample population. Transient fault classification accuracy was observed to increase to near 100\%. However, intermittent classification accuracy also increased at the same time, for cases with and without transient noise.

\begin{table}[h]

	\captionsetup{justification=centering}
\begin{tabular}{ccccc}
\hline
\multirow{3}{*}{Circuit} & \multicolumn{4}{c}{Accuracy (\%)}                                                                 \\ \cline{2-5} 
                         & \multirow{2}{*}{Cross-validation} & \multicolumn{2}{c}{Intermittent} & \multirow{2}{*}{Transient} \\ \cline{3-4}
                         &                                   & w/o noise      & with noise      &                            \\ \hline
p45k                     & 84.46                             & 68.75          & 60.41           & 100                        \\
p100k                    & 80.00                             & 81.39          & 52.38           & 100                        \\
p141k                    & 80.04                             & 78.57          & 73.80           & 100                        \\
p267k                    & 75.17                             & 75.00          & 55.00           & 100                        \\
p279k                    & 83.41                             & 81.57          & 52.63           & 91.83                      \\
p295k                    & 92.28                             & 93.02          & 86.04           & 91.83                     \\
\hline
\end{tabular}
\caption {Classification accuracy for sigmoid kernel, without permanent faults}
\label{tab:sigwop}
\end{table}

\subsection{Analysis of intermittent fault classification}
\section{With class-weight optimization}
\section{Other possibilities}
\subsection{Using single training set for prediction}
\subsection{Using known model to extrapolate results}
\section{Impact of yield and quality of final product}
